\documentclass[a4paper]{memoir}

\usepackage[english]{babel} % load typographical rules for the english language
\usepackage{graphics} % for \scalebox
\usepackage{hyperref} % for \href

% styling
\setsecnumdepth{subsubsection} % how deep to number sections
\setlength{\parindent}{0em} % horizontal indent for first line of paragraph
\setlength{\parskip}{1em} % vertical space between paragraphs

\title{Report Writing \\ \scalebox{0.85}{for Software BSc and MSc Projects}}
\author{Aslak Johansen}

\begin{document}

\maketitle
\tableofcontents

%%%%%%%%%%%%%%%%%%%%%%%%%%%%%%%%%%%%%%%%%%%%%%%%%%%%%%%%%%%%%%%%%%%%%%%%%%%%%%%%%
%%%%%%%%%%%%%%%%%%%%%%%%%%%%%%%%%%%%%%%%%%%%%%%%%%%%%%%%%%%%%%%%%%%% Introduction
\chapter{Introduction}

% purpose

\section{Supervisor}

\subsection{Role}

\subsection{Interaction}

\subsection{Choice}

% intro: partnership
The relationship between the students of a group, their supervisor(s) is a partnership. Ideally, all partners should get something out of it. From the supervisors side that will usually be a domain or field of interest, and the project could be rooted in a research project.

% if it doesnt work out
If you have a negative previous experience with a potential supervisor (e.g., from a bachelor project), then you should obviously look elsewhere. Do note that this is does not have to indicate more that incompatibilities of personality types.

% reasons
Good reasons to choose a supervisor may include:
\begin{itemize}
  \item You like the supervisor on a personal level. Do note that supervision is not a personal relationship.
  \item You have enjoyed classes taught by the supervisor. Do note that supervision style may differ from teaching style.
  \item Overlap in technological interests.
  \item Overlap in domain interests.
\end{itemize}

\section{Group Composition}

% project length

% don't do it alone
Groups of one can work, but they rarely do. What usually happens is that -- lacking experience with a project of this size -- the student starts postponing tasks (e.g., due to waning motivation). Groups of two are much more robust in that sense, likely due to a combination of a feeling of responsibility towards the other group member and the shared support structure. Groups of three can also work, but here the trick becomes (i) to size to project workload, and (ii) to make it clear that all have done their fair share of the project.

% friends but not too good friends

\section{Project Choice}

\section{Project Size}
\label{sec:projectsize}

\section{Industrial Collaboration}

% positive

% seen through the eyes of the company

% negative

\subsection{NDA}

% purpose and involvement
Often, an industrial collaborator will insist on the anyone coming into contact with what they consider company secrets to sign a nondisclosure agreement (NDA). That includes the group members and SDU. SDU will sign this on behalf of the supervisor(s) and censor. You can get the ball rolling by contacting \href{mailto:contracts@sdu.dk}{contracts@sdu.dk}.

% warning
You should, however, be aware that getting an NDA in place can be a lengthy process. All parties need to agree on the legalese and that means that representatives of the legal organizations have to dead through the document and agree on the contents. SDU has a standard NDA, but the industrial partner may not be happy about it. Expect this process to be measured in months, and expect not to have access to material from the industrial collaborators before it is in place. If you are doing a 40 ECTS project that is likely to be highly inconvenient, but if you are doing a 30 ECTS project that can quickly become detrimental. The lesson learned is to start as early as possible, and aim to have that NDA in place \textsl{before} handing in the project description.

\section{Deadlines}

At SDU there are a few deadlines to be concerned with. These are:
\begin{itemize}
  \item The deadline for signing up for a 40 ECTS MSc project and is, naturally, only applicable to MSc projects. See section \ref{sec:projectsize} for details.
  \item The deadline of handing in the project description. Please note that this needs to be approved. See section \ref{sec:projectdesc} for details.
  \item Usually (but not always) there is a poster session approx 1 month before handin of the final report. It may be mandatory for you and it may be optional.
  \item The deadline of handing in the final report.
\end{itemize}

%%%%%%%%%%%%%%%%%%%%%%%%%%%%%%%%%%%%%%%%%%%%%%%%%%%%%%%%%%%%%%%%%%%%%%%%%%%%%%%%%
%%%%%%%%%%%%%%%%%%%%%%%%%%%%%%%%%%%%%%%%%%%%%%%%%%%%%%%%%%%%% Project Description
\chapter{Project Description}
\label{sec:projectdesc}

%%%%%%%%%%%%%%%%%%%%%%%%%%%%%%%%%%%%%%%%%%%%%%%%%%%%%%%%%%%%%%%%%%%%%%%%%%%%%%%%%
%%%%%%%%%%%%%%%%%%%%%%%%%%%%%%%%%%%%%%%%%%%%%%%%%%%%%%%%%%%%%%%%%%%%% First Steps
\chapter{First Steps}

\section{Process}

\section{Working Document}

\section{Report Structure}

%%%%%%%%%%%%%%%%%%%%%%%%%%%%%%%%%%%%%%%%%%%%%%%%%%%%%%%%%%%%%%%%%%%%%%%%%%%%%%%%%
%%%%%%%%%%%%%%%%%%%%%%%%%%%%%%%%%%%%%%%%%%%%%%%%%%%%%%%%%%%%%% The Art of Writing
\chapter{The Art of Writing}

%%%%%%%%%%%%%%%%%%%%%%%%%%%%%%%%%%%%%%%%%%%%%%%%%%%%%%%%%%%%%%%%%%%%%%%%%%%%%%%%%
%%%%%%%%%%%%%%%%%%%%%%%%%%%%%%%%%%%%%%%%%%%%%%%%%%%%%%%%%%%%%%% Literature Review
\chapter{Literature Review}

%%%%%%%%%%%%%%%%%%%%%%%%%%%%%%%%%%%%%%%%%%%%%%%%%%%%%%%%%%%%%%%%%%%%%%%%%%%%%%%%%
%%%%%%%%%%%%%%%%%%%%%%%%%%%%%%%%%%%%%%%%%%%%%%%%%%%%%%%%%%%%%%%%%%%%%% Evaluation
\chapter{Evaluation}

\end{document}
