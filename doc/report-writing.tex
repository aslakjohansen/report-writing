\documentclass[a4paper, oneside]{memoir}

\usepackage[english]{babel} % load typographical rules for the english language
\usepackage{graphics} % for \scalebox
\usepackage{hyperref} % for \href

% https://github.com/latex3/babel/issues/51
\makeatletter\AtBeginDocument{\let\@elt\relax}\makeatother

% styling
\setsecnumdepth{subsubsection} % how deep to number sections
\setlength{\parindent}{0em} % horizontal indent for first line of paragraph
\setlength{\parskip}{1em} % vertical space between paragraphs

\newcommand{\textdesc}[1]{\textit{\textbf{#1}}}
\newcommand{\descitem}[1]{\item \textdesc{#1}}

\title{Report Writing \\ \scalebox{0.85}{for Software BSc and MSc Projects}}
\author{Aslak Johansen}

\begin{document}

\maketitle

\setcounter{tocdepth}{2}
\tableofcontents

%%%%%%%%%%%%%%%%%%%%%%%%%%%%%%%%%%%%%%%%%%%%%%%%%%%%%%%%%%%%%%%%%%%%%%%%%%%%%%%%
%%%%%%%%%%%%%%%%%%%%%%%%%%%%%%%%%%%%%%%%%%%%%%%%%%%%%%%%%%%%%%%%%%% Introduction
\chapter{Introduction}

% purpose

\section{Supervisor}

First of all: All supervisors are different. If your supervisor disagrees with this section, then they are right and this section is wrong.

\subsection{Role}

% inclusion: report structure (when presented with text of reasonable quality)

% exclusion: basic grammar, spelling

\subsection{Interaction}

The perfect interaction model for supervision depends on the specific project. However, as a starting point I like to have a weekly recurring status meeting of 30 minutes that goes into the calendar. That way there is time allocated. If I don't receive an agenda over Discord 24 hours before the meeting I consider it canceled and may reallocate the time. In some weeks there won't be anything to cover in this status meeting, and in other weeks there may be a need for an extra meeting.

When it comes to the day-to-day supervision (aka the small questions and messages) I find that a per-project Discord channels works best. That setup also integrates well with online meetings and screensharing.

\subsection{Choice}

% intro: partnership
The relationship between the students of a group, their supervisor(s) is a partnership. Ideally, all partners should get something out of it. From the supervisors side that will usually be a domain or field of interest, and the project could be rooted in a research project.

% if it doesnt work out
If you have a negative previous experience with a potential supervisor (e.g., from a bachelor project), then you should obviously look elsewhere. Do note that this is does not have to indicate more that incompatibilities of personality types.

% reasons
Good reasons to choose a supervisor may include:
\begin{itemize}
  \item You like the supervisor on a personal level. Do note that supervision is not a personal relationship.
  \item You have enjoyed classes taught by the supervisor. Do note that supervision style may differ from teaching style.
  \item Overlap in technological interests.
  \item Overlap in domain interests.
\end{itemize}

\section{Project Size}
\label{sec:projectsize}

% project length
Currently, at SDU, a BSc project is 15 ECTS points across the spring semester, and a MSc project is either 30 ECTS across the spring semester or 40 ECTS points across two semesters starting in the fall. The workload of the 40 ECTS variant is distributed so that 10 ECTS points are placed in the fall semester and the remaining 30 ECTS points fill up the entire spring semester. When that model is followed, the fall semester is typically spent interviewing stakeholders, doing the literature review and generally laying things out so that everything is ready for a focused spring semester. If you have found a project that gets you excited, and you should, then I really recommend that you choose a 40 ECTS point MSc. The 30 ECTS points variant has a tendency to get crammed. Be aware that there is a deadline for applying for a 40 ECTS project.

\section{Group Composition}

% don't do it alone
Groups of one can work, but they rarely do. What usually happens is that -- lacking experience with a project of this size -- the student starts postponing tasks (e.g., due to waning motivation). Groups of two are much more robust in that sense, likely due to a combination of a feeling of responsibility towards the other group member and the shared support structure. Groups of three can also work, but here the trick becomes (i) to size to project workload, and (ii) to make it clear that all have done their fair share of the project.

% friends but not too good friends
It is important for group members to be on good terms. Positive experience from previous groupwork is a plus. However, groups consisting of too \textsl{good} friends have a tendency to find topics irrelevant to the project to spend their time on. While that is a healthy thing it is not conductive in terms of progressing the project.

\section{Project Choice}

\section{Industrial Collaboration}

% positive

% seen through the eyes of the company

% negative

\subsection{Non-Disclosure Agreement}

% purpose and involvement
Often, an industrial collaborator will insist on the anyone coming into contact with what they consider company secrets to sign a nondisclosure agreement (NDA). That includes the group members and SDU. SDU will sign this on behalf of the supervisor(s) and censor. You can get the ball rolling by contacting \href{mailto:contracts@sdu.dk}{contracts@sdu.dk}. But before doing so, you should read their intro\footnote{\url{https://www.sdu.dk/en/om_sdu/institutter_centre/tekinnovation/for_studerende/nda}}.

% warning
You should, however, be aware that getting an NDA in place can be a lengthy process. All parties need to agree on the legalese and that means that representatives of the legal organizations have to dead through the document and agree on the contents. SDU has a standard NDA, but the industrial partner may not be happy about it. Expect this process to be measured in months, and expect not to have access to material from the industrial collaborators before it is in place. If you are doing a 40 ECTS project that is likely to be highly inconvenient, but if you are doing a 30 ECTS project that can quickly become detrimental. The lesson learned is to start as early as possible, and aim to have that NDA in place \textsl{before} handing in the project description.

\section{Deadlines}

At SDU there are a few deadlines to be concerned with. These are:
\begin{itemize}
  \item The deadline for signing up for a 40 ECTS MSc project and is, naturally, only applicable to MSc projects. See section \ref{sec:projectsize} for details.
  \item The deadline of handing in the project description. Please note that this needs to be approved. See section \ref{sec:projectdesc} for details.
  \item Usually (but not always) there is a poster session approx 1 month before handin of the final report. It may be mandatory for you and it may be optional.
  \item The deadline of handing in the final report.
\end{itemize}

%%%%%%%%%%%%%%%%%%%%%%%%%%%%%%%%%%%%%%%%%%%%%%%%%%%%%%%%%%%%%%%%%%%%%%%%%%%%%%%%
%%%%%%%%%%%%%%%%%%%%%%%%%%%%%%%%%%%%%%%%%%%%%%%%%%%%%%%%%%%% Project Description
\chapter{Project Description}
\label{sec:projectdesc}

%%%%%%%%%%%%%%%%%%%%%%%%%%%%%%%%%%%%%%%%%%%%%%%%%%%%%%%%%%%%%%%%%%%%%%%%%%%%%%%%
%%%%%%%%%%%%%%%%%%%%%%%%%%%%%%%%%%%%%%%%%%%%%%%%%%%%%%%%%%%%%%%%%%%% First Steps
\chapter{First Steps}

\section{Tools}

% google docs
Many people like to work in Google Docs, and it truly is convenient for many things. However, the typographical qualities of the resulting PDF are awful. After having seen a a great many reports written in Google Docs, I have reached the conclusion that it is likely impossible to create a remotely professional looking report in Google Docs.

% microsoft word
Microsoft Word is an alternative. I have heard many students say that it is possible to create a professionally looking report in Microsoft Word, and they have all failed to do so. I am starting to think that it is possible to get quite close, but that the effort required is ridiculously high. The same goes for Libre/Open Office.

% latex
\LaTeX has all the right typographical rules needed, and it will allow you to tweak everything to your hearts desire. But, crucially, you don't have to. If you don't want to know about these things, then just use the (generally) sensible defaults, and you are already ahead.

% working with latex
\LaTeX is written as code in plain text files, and can therefore easily live in a git repository. There are online \LaTeX editors (e.g., Overleaf) that can be linked to such a git repository. But do \underline{not} rely on them. Their uptime is not impressive.

% linking to code
Finally, there are benefits to having the report code either in the same repository as the project code, or in one that is checked out at a well defined path relative to the code repository. This allows for inclusions and automatic building of the report (e.g., to include results of tests or code snippets).

% required features
Generally speaking I would expect support for the following in an even remotely modern system for writing IT related reports:
\begin{itemize}
  \item Inclusion of external standardized vector graphics formats\footnote{Note that many vector graphics formats support the inclusion of bitmapped graphics. Doing so to a screenshot does not make the result vector graphics.} (e.g., PDF and SVG) without going through rasterization.
  \item Support for ligatures.
  \item Proper support for kerning.
  \item Automatic inclusion and highlight of code.
  \item Floats.
  \item References that "just work".
  \item Citation system.
  \item Automatic table of contents.
\end{itemize}

% why the focus
Why does it matter? First of all, it shows that you care. If you did not care very much about this report, why should the reader? Secondly, it is obvious whether or not you used the right tools. If you don't command the right tools as an engineer, that's a problem right there!

\section{Process}

\section{Working Document}

% This file represents a dissertation. In the beginning of it you are likely to have a central thesis, or maybe a number of theses that you wish to explore.

\section{Report Structure}

%%%%%%%%%%%%%%%%%%%%%%%%%%%%%%%%%%%%%%%%%%%%%%%%%%%%%%%%%%%%%%%%%%%%%%%%%%%%%%%%
%%%%%%%%%%%%%%%%%%%%%%%%%%%%%%%%%%%%%%%%%%%%%%%%%%%%%%%%%%%%% The Art of Writing
\chapter{The Art of Writing}

\section{Commandments}

\begin{enumerate}
  \descitem{Make the Effort} % take pride in your work
    As with pretty much else in life, if you want to excel at writing you need to put in significant amounts of focused work and active considerations about the nature of what makes a text good. Take pride in your work.
  \descitem{It Has to Make Sense}
  \descitem{Give and Take Credit}
  \descitem{The Good Practice of Selective Violation of Best Practices / Be Prepared to Violate a Best Practice or Two/Don't Blindly Follow Best Practices / Be Critical Towards Best Practices / Challenge Best Practices}
  \descitem{Do not Underestimate Experimental Evaluation}
  \descitem{Respect your/the Audience}
  \descitem{One message One sentence}
  \descitem{A sentence should be self-sustaining}
  \descitem{The Specific-Generic Spectrum} % Can we make this statement more specific to a degree where it is as wide as possible without becoming untrue for some cases?
  \descitem{Make the argument explicit}
    When providing an argument, do not rely on the reader to fold out steps that you consider implicit. The reader may have other standards for what is implicit, or have a different cultural background where those steps are not obvious. Also, this is an opportunity to show that you can construct a line of reasoning, and that opportunity should be grabbed.
  \descitem{Remember the Context} % E.g., the word "technical" can be rooted in software, but could also be legal or mechanical. I assume that people from legal also refer to the technical aspects, but have a fairly different interpretation of what that means.
\end{enumerate}

\section{Pieces}

\subsection{Definitions}

\section{Questions}

\begin{itemize}
  \descitem{From a scientific point of view -- why does it matter?}
\end{itemize}

\section{Anti-Patterns}

\subsection{Use-Before-Definition}

\subsection{Reason-Therefore-Reason}

% Because of the "Therefore" you already have a reason and that reason will appear -- to the reader -- to be in conflict with what comes after the "as" of the last sentence.

\subsection{Sentence Repetition}

% First introducing a general statement, and then providing two examples

\subsection{Two Paragraphs for Single Purpose}

% This paragraph serves exactly the same purpose as the previous one, and as the saying goes: There can be only one!

\subsection{Should Without a Question}

% Leading with a "should" tells the reader that you are asking a question. If you flip it around (i.e., research should) makes a concluding statement.

%%%%%%%%%%%%%%%%%%%%%%%%%%%%%%%%%%%%%%%%%%%%%%%%%%%%%%%%%%%%%%%%%%%%%%%%%%%%%%%%
%%%%%%%%%%%%%%%%%%%%%%%%%%%%%%%%%%%%%%%%%%%%%%%%%%%%%%%%%%%%%%%%%%% Introduction
\chapter{Introduction}

%%%%%%%%%%%%%%%%%%%%%%%%%%%%%%%%%%%%%%%%%%%%%%%%%%%%%%%%%%%%%%%%%%%%%%%%%%%%%%%%
%%%%%%%%%%%%%%%%%%%%%%%%%%%%%%%%%%%%%%%%%%%%%%%%%%%%%%%%%%%%%% Literature Review
\chapter{Literature Review}

%%%%%%%%%%%%%%%%%%%%%%%%%%%%%%%%%%%%%%%%%%%%%%%%%%%%%%%%%%%%%%%%%%%%%%%%%%%%%%%%
%%%%%%%%%%%%%%%%%%%%%%%%%%%%%%%%%%%%%%%%%%%%%%%%%%%%%%%%%%%%%%%%%%% Requirements
\chapter{Requirements}

%%%%%%%%%%%%%%%%%%%%%%%%%%%%%%%%%%%%%%%%%%%%%%%%%%%%%%%%%%%%%%%%%%%%%%%%%%%%%%%%
%%%%%%%%%%%%%%%%%%%%%%%%%%%%%%%%%%%%%%%%%%%%%%%%%%%%%%%%%%%%%%%%%%%%%%% Analysis
\chapter{Analysis}

%%%%%%%%%%%%%%%%%%%%%%%%%%%%%%%%%%%%%%%%%%%%%%%%%%%%%%%%%%%%%%%%%%%%%%%%%%%%%%%%
%%%%%%%%%%%%%%%%%%%%%%%%%%%%%%%%%%%%%%%%%%%%%%%%%%%%%%%%%%%%%%%%%%%%%%%%% Design
\chapter{Design}

%%%%%%%%%%%%%%%%%%%%%%%%%%%%%%%%%%%%%%%%%%%%%%%%%%%%%%%%%%%%%%%%%%%%%%%%%%%%%%%%
%%%%%%%%%%%%%%%%%%%%%%%%%%%%%%%%%%%%%%%%%%%%%%%%%%%%%%%%%%%%%%%%% Implementation
\chapter{Implementation}

%%%%%%%%%%%%%%%%%%%%%%%%%%%%%%%%%%%%%%%%%%%%%%%%%%%%%%%%%%%%%%%%%%%%%%%%%%%%%%%%
%%%%%%%%%%%%%%%%%%%%%%%%%%%%%%%%%%%%%%%%%%%%%%%%%%%%%%%%%%%%%%%%%%%%% Evaluation
\chapter{Evaluation}

% intro:
When you have constructed your system, you have certainly achieved something. But at that point we know very little about \textsl{what} you have achieved. Does it work correctly? How well does it meet the requirements? To be able to make claims in these regards, you need an evaluation. What and how you need to evaluate depends on which kinds of claims you wish to make. At the end of the day the claims supported by your evaluation will define the success of the system. From a report perspective, that is.

% TODO: needs a rewrite:
My recommendation is for you to think the evaluation into the design so that you end up designing for evaluation. One benefit of that would be that your mental model of the evaluation process evolves along with the project. Another would be that that you end up having natural points for instrumentation in the codebase that gives you exactly the kind of access that you need for your *ideal* evaluation.

\section{Criteria}

\subsection{Correctness}

\subsection{Usability}

\subsection{Performance}

\subsection{Availability}

\subsection{Scalability}

\section{Methods}

% argument: what is

\subsection{Textual Argument}

\subsection{Performance Evaluation}

\subsection{User Test}

\section{Process}

Go through the requirements; one at a time. Determine the best way of evaluating this particular requirement. 

\begin{figure}[tbp]
  contents
  \caption{Method suitability for various criteria.}
\end{figure}

\section{Presentation}

%%%%%%%%%%%%%%%%%%%%%%%%%%%%%%%%%%%%%%%%%%%%%%%%%%%%%%%%%%%%%%%%%%%%%%%%%%%%%%%%
%%%%%%%%%%%%%%%%%%%%%%%%%%%%%%%%%%%%%%%%%%%%%%%%%%%%%%%%%%%%%%%%%%%%% Discussion
\chapter{Discussion}

%%%%%%%%%%%%%%%%%%%%%%%%%%%%%%%%%%%%%%%%%%%%%%%%%%%%%%%%%%%%%%%%%%%%%%%%%%%%%%%%
%%%%%%%%%%%%%%%%%%%%%%%%%%%%%%%%%%%%%%%%%%%%%%%%%%%%%%%%%%%%%%%%%%%% Future Work
\chapter{Future Work}

%%%%%%%%%%%%%%%%%%%%%%%%%%%%%%%%%%%%%%%%%%%%%%%%%%%%%%%%%%%%%%%%%%%%%%%%%%%%%%%%
%%%%%%%%%%%%%%%%%%%%%%%%%%%%%%%%%%%%%%%%%%%%%%%%%%%%%%%%%%%%%%%%%%%%% Conclusion
\chapter{Conclusion}

\end{document}
