\section{Actor Model}

% actor

% message passing

% share nothing architecture

\subsection{Supervision Trees}

% failures, sandbox, crashes, monitoring
Actors, like any other construct of logic, can fail for reasons ranging from common (e.g., a bug) to exotic (e.g., random bitflips\idx{Bitflip}). To deal with this, actors should operate within some sort of sandbox\idx{Sandbox}. Still, crashes\idx{Crash} will occur and, when that happens, a part of the logic will go away. This raises to need for monitoring\idx{Monitoring} actors, detecting their crashes and action on them by starting up a replacement actor.

% supervisor role: what to do when there is a crash? need for a component to handle this event, lifecycle management (starting, stopping, dependency aware)
This is part of the responsibility of a \textsl{supervisor}\idx{Supervisor} actor. A supervisor is a class of actors that is in charge of lifecycle\idx{Lifecycle management} management of other actors (called children). It knows how to start them, how to stop them (gracefully or otherwise), and it is aware of the dependencies between these supervised actors. When a crash of a \textsl{supervised} process happens, its supervisor will detect this, and act accordingly. This involves stopping depending actors, and starting the actors up again in the necessary order. In short, a supervisor is an actor, that supervises a number of other actors in order to make sure that the overall functionality will \textsl{heal}\idx{Self-healing} in case of a failure.

% supervision tree

% figure: supervisor tree

