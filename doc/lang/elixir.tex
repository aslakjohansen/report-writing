{
\setmonofont[
  Contextuals={Alternate}
]{Fira Code}

\section{Elixir}

\idx{Elixir}A distributed language that is built around the actor model with a focus on concurrency\idx{Concurrency} and availability\idx{Availability}. It is a modern language with a long list of attractive features. Its main downside is that it is not a numerically fast language.

\subsection{Virtual Machine}

% beam
Back in the 80's the only place where programmers created distributed systems were telephone\idx{Telephone exchange system} exchanges. In this domain, a need arose for new platforms with properties of (i) distribution, (ii) soft real-time properties, and (iii) close to zero downtime, even across software updates and hardware faults. Ericsson\idx{Ericsson} -- backed by monopoly conditions and the IT hub of Kista -- developed a platform to meet these needs. It is -- by far -- the most battle tested platform in this space. The platform consists of:
\begin{itemize}
  \item The Erlang\idx{Erlang} programming language.
  \item The BEAM\idx{BEAM} (distributed) virtual machine.
  \item The Open\idx{OTP} Telecom Platform (OTP).
\end{itemize}

% who uses beam
A number of programming languages compile to BEAM code. Among those, Elixir. However, BEAM is not just a virtual machine for yet another group of small languages. BEAM (e.g., Erlang and Elixir) has a few users, including: 
\begin{itemize}
  \descitem{Discord}\idx{Discord} is using Elixir to scale to 5M concurrent users.
%        \\ \url{https://discord.com/blog/how-discord-scaled-elixir-to-5-000-000-concurrent-users}
  \descitem{Facebook}\idx{Facebook} uses Erlang to power its chat service serving 100M active users.
%        \\ \url{https://kyan.com/news/an-introduction-to-the-elixir-programming-language}
  \descitem{WhatsApp}\idx{WhatsApp} uses Erlang to run messaging servers, each covering 2M users.
  \descitem{Ericsson} uses Erlang for its GPRS, 3G, 4G and 5G infrastructure, and has a market share of 40\%.
  \item In 2018 \textbf{Cisco}\idx{Cisco} shipped 2M devices running Erlang. At that point 90\% of all internet traffic went through Erlang based nodes.
  \item 100+ NEPs\idx{NEP} (network equipment providers) -- including the top 8 -- use Erlang based components in their products. Same numbers for SPs\idx{Service provider} (service providers).
%        \\ \url{https://codesync.global/media/https-youtu-be-077-xjv6plq/}
\end{itemize}

\subsection{Execution}
\subsubsection{Compilation}
\subsubsection{Interpretation}

\begin{minted}[breaklines,fontsize=\tiny]{iex}
$ iex
Erlang/OTP 24 [erts-12.2.1] [source] [64-bit] [smp:8:8] [ds:8:8:10] [async-threads:1] [jit]

Interactive Elixir (1.12.2) - press Ctrl+C to exit (type h() ENTER for help)
iex(1)> 1+1
2
iex(2)> incr = fn i -> i + 1 end
#Function<44.65746770/1 in :erl_eval.expr/5>
iex(3)> incr.(1)
2
\end{minted}

\subsubsection{LiveBook}

\idx{Livebook}
\begin{itemize}
  \item Get LiveBook: \url{https://livebook.dev}
  \item Watch demos: \url{https://www.youtube.com/@livebookdev}
\end{itemize}

\subsection{Functions}

\subsection{Modules}

\inputminted[fontsize=\normalsize]{elixir}{../src/elixir/fibonacci.ex}

\begin{minted}[breaklines]{iex}
iex(1)> Enum.map(0..10, fn n -> Fibonacci.fib(n) end)
[0, 1, 1, 2, 3, 5, 8, 13, 21, 34, 55]
\end{minted}

\subsection{Pattern Matching}

\subsection{Pipe Operator}

\subsection{GenServer}

\subsection{Registry}

\subsection{Cache}

\subsection{Supervision}
\subsubsection{Supervisor}
\subsubsection{DynamicSupervisor}

\subsection{GenStage}
\subsubsection{Flow}

\idx{Flow}\url{https://github.com/dashbitco/flow/}

\subsubsection{Broadway}

\idx{Broadway}\url{https://elixir-broadway.org}

\subsection{Domains}

\subsubsection{Web}

% phoenix
\idx{Phoenix}\url{https://www.phoenixframework.org}

% channels

% liveviews

\subsubsection{Data Science}

\url{https://www.youtube.com/@titantech2271/videos}

\subsubsection{Internet of Things}

\idx{Nerves}\url{https://nerves-project.org}

}
