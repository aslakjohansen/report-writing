\section{Other Languages}

\begin{itemize}
  \descitem{Elm}\footnote{\url{https://elm-lang.org}}\idx{Elm programming language} A \quoted{delightful} browser language with a focus on simplicity, reliability and (to some degree) speed.
  \descitem{Julia}\footnote{\url{https://julialang.org}}\idx{Julia programming language}
  \descitem{Mojo}\footnote{\url{https://www.modular.com/mojo}}\idx{Mojo programming language} An ungoing attempt at combinin Python with the performance of C. It is currently a work in progress, but will (the developers hope) include zero-cost abstractions, borrow checker, and python compatibility.
  \descitem{R}\footnote{\url{https://www.r-project.org}}\idx{R programming language} R is designed for statistical computing. It has a syntax that takes some getting used to and gives you access to the ggplot2\footnote{\url{https://ggplot2.tidyverse.org}}\idx{Ggplot2} plotting package.
  \descitem{Raku}\footnote{\url{https://raku.org}}\idx{Raku programming language} Raku used to be known as Perl\footnote{Perl, or the \textsl{swiss army chainsaw}, was a popular scripting language with the strongest regular expression engine in the business (before Raku, that is). Then Python took over due to it's simplicity.}\idx{Perl 6} 6. Raku introduces grammars as a first-class citizen, has optional and gradual typing, supports both object-orientation and functional paradigms, and supports both parallelism and concurrency.
\end{itemize}
