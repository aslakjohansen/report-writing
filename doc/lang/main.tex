\chapter{Programming Languages}

\begin{inspiration}{Alan J. Perlis\idxx{Perlis, Alan}{Alan Perlis}\cite{10.1145/947955.1083808}}
  \quoted{A language that doesn't affect the way you think about programming, is
not worth knowing}
\end{inspiration}

\begin{inspiration}{Bjarne Stroustrup\idxx{Stroustrup, Bjarne}{Bjarne Stroustrup}\cite{10.5555/2543987}}
  \quoted{There are only two kinds of languages: the ones people complain about and the ones nobody uses}
\end{inspiration}

% numerical performance: definition, the petaflops club

\section{Properties}

\subsection{Memory Management}

\subsection{Typing}

\subsection{Homoiconicity}

\section{Getting Started}

Projects for learning a new language:
\begin{itemize}
  \item John\idxx{Conway, John Horton}{John Conway} Conway's Game of Life\footnote{\url{https://en.wikipedia.org/wiki/Conway\%27s_Game_of_Life}}\idx{Game of Life}\cite{Gardener1970MathematicalGT}.
    \begin{itemize}
      \item Initialization by either random, or file loading.
      \item Flexibility allowing for alternative rule-set.
      \item Hex-tiling as an alternative to the classic grid layout.
      \item Visualizations (e.g., ASCII art)?
    \end{itemize}
  \item HTTP/FTP/MQTT server on top of plain sockets.
  \item Sudoku
    \begin{itemize}
      \descitem{Generator} Generate puzzles under consideration of difficulty, symmetry and ensuring a single solution.
      \descitem{Persistence} Marshall\idx{Marshalling} the datastructure and store it to a file. Then read and unmarchall it again.
      \descitem{Pretty-Printing} Code that can \textsl{pretty-print}\idx{Pretty-printing} either using ASCII art or to PDF.
      \descitem{Solver}\idx{Solver} Code that can solve a puzzle.
    \end{itemize}
  \item Sieve of Eratosthenes\footnote{\url{https://en.wikipedia.org/wiki/Sieve_of_Eratosthenes}}. Real programmers know whether their social security number is prime!\footnote{Should your social security number be even, you could find the largest prime below this number.}
  \item A solver for the traveling salesman problem\footnote{\url{https://en.wikipedia.org/wiki/Travelling_salesman_problem}}
\end{itemize}

\section{Make}

\section{Assembly}

\section{C}

\idx{C}Low-level language that adds syntax for the fundamental\idx{Fundamental abstractions} abstractions in data and logic to assembly. This makes it a simple language to learn, and just as simple a language to mess up in. It has a precompiler that is very convenient for producing software product lines\idx{Product line}.

\subsection{Precompiler}
\subsection{Pointer Arithmetic}


\section{Elixir}

\idx{Elixir}A distributed language that is built around the actor model with a focus on concurrency\idx{Concurrency} and availability\idx{Availability}. It is a modern language with a long list of attractive features. Its main downside is that it is not a numerically fast language.

\subsection{Execution}
\subsubsection{Compilation}
\subsubsection{Interpretation}
\subsubsection{LiveBook}

\subsection{Functions}

\subsection{Pattern Matching}

\subsection{Pipe Operator}

\subsection{GenServer}

\subsection{Registry}

\subsection{Cache}

\subsection{Supervision}
\subsubsection{Supervisor}
\subsubsection{DynamicSupervisor}

\subsection{GenStage}
\subsubsection{Flow}

\idx{Flow}\url{https://github.com/dashbitco/flow/}

\subsubsection{Broadway}

\idx{Broadway}\url{https://elixir-broadway.org}

\subsection{Domains}

\subsubsection{Web}

% phoenix
\idx{Phoenix}\url{https://www.phoenixframework.org}

% channels

% liveviews

\subsubsection{Data Science}

\url{https://www.youtube.com/@titantech2271/videos}

\subsubsection{Internet of Things}

\idx{Nerves}\url{https://nerves-project.org}


\section{Go}

\idx{Go}\index{Go-lang} Go is a fast language focusing on concurrency\idx{Concurrency} under a single operating system (i.e., vertical scalability\idxx{Scalability!Vertical}{Vertical scalability}). The creators are opinionated, and you are unlike to agree on all of their opinions. But there's always the highway. Go programmers are known as \textsl{gophers}\idx{Gopher}.

% link to youtube video
Rob Pike's\idxx{Pike, Rob}{Rob Pike} \quoted{Concurrency is not Parallelism} presentation is a good teaser for the language\footnote{While his examples are in Go, the same principles apply to Elixir (see section \ref{lang:elixir}).}: \url{https://www.youtube.com/watch?v=oV9rvDllKEg}

\subsection{Go Routines}
\subsection{Channels}


\section{Python}

\idx{Python}Script\idx{Script Language} language which wont bite too much. It won't do concurrency\footnote{Unless you \textsl{\textbf{really}} know what you are doing, and 95\% of those who think they do don't.}, syntax errors are produced at runtime, and it is numerically pretty slow. If you can accept this (e.g., if you are doing script work) it's a pretty good language.

\subsection{Versions}

\subsection{Embedded Systems}

\idxx{Python!Embedded systems}{Embedded systems}As a stop-the-world\idxx{Garbage collection!Stop-the-world}{Stop-the-world Garbage collection} garbage collected language, Python is inherently problematic for embedded applications. To ensure successful communication with peripheral devices (e.g., radios, flash chips and digital sensors) that employ time-critical communication protocols\idxx{Communication!Time critical}{Time-critical communicaiton}, one is likely to have to either disable garbage collection during such operations or force it just before. This will pollute the codebase, introduce potential and hard-to-find bugs, and result in an efficiency penalty.

MicroPython\idx{MicroPython}\footnote{\url{https://micropython.org}} is an implementation of Python 3 with microcontrollers in mind. It is efficient for quick and dirty prototyping, but shouldn't be used for production purposes.

\subsection{Embedding}

% purpose: access to python modules, plugin/scripting language

% c api
\idxx{Python!Embedding}{Embedding}\url{https://docs.python.org/3/c-api/intro.html}

% binding to other languages (e.g., go)

\subsection{Symbolic Mathematics}

\idx{SymPy}\url{https://www.sympy.org}

\subsection{Object Orientation}

\subsection{AsyncIO}


\section{Rust}

\idx{Rust}Rust is a modern C-replacement language that focuses on safety (both memory and thread) and speed. Rust programmers are known as \textsl{rustaceans}.

\subsection{Drop-In Replacement for C}
\subsection{Zero-Cost Abstraction}
\subsection{Ownership Model}


\section{Julia}
\label{lang:julia}

Julia\footnote{\url{https://julialang.org}}\idx{Julia programming language} is a dynamically typed high-performance language that supports metaprogramming. It has both a compiled and an interactive mode, and it is also the \say{ju} in \say{jupyter}\idx{Jupyter}.

\subsection{Units}

\idxx{Unitful.jl package@\texttt{Unitful.jl} package}{\texttt{Unitful.jl} package}\url{https://painterqubits.github.io/Unitful.jl/stable/}

Example:
\begin{minted}[breaklines]{julia}
julia> using Unitful; using Unitful: kg, g, km, hr, minute

julia> 1u"kg" == 1000u"g"
true

julia> uconvert(km/hr, 5.03u"km"/22.7u"minute")
13.295154185022028 km hr^-1
\end{minted}


\section{Other Languages}

\begin{itemize}
  \descitem{Elm}\footnote{\url{https://elm-lang.org}}
  \descitem{Julia}\footnote{\url{https://julialang.org}}
  \descitem{Mojo}\footnote{\url{https://www.modular.com/mojo}}
  \descitem{R}\footnote{\url{https://www.r-project.org}} R is for statistical computing. It has a syntax that takes some getting used to and gives you access to the ggplot2\footnote{\url{https://ggplot2.tidyverse.org}} plotting package.
  \descitem{Raku}\footnote{\url{https://raku.org}}
\end{itemize}


