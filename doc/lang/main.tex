\chapter{Programming Languages}

\section{Properties}

\subsection{Memory Management}

\subsection{Typing}

\section{Getting Started}

Projects for learning a new language:
\begin{itemize}
  \item John\idxx{Conway, John Horton}{John Conway} Conway's Game of Life\footnote{\url{https://en.wikipedia.org/wiki/Conway\%27s_Game_of_Life}}\idx{Game of Life}\cite{Gardener1970MathematicalGT}.
    \begin{itemize}
      \item Initialization by either random, or file loading.
      \item Flexibility allowing for alternative rule-set.
      \item Hex-tiling as an alternative to the classic grid layout.
      \item Visualizations (e.g., ASCII art)?
    \end{itemize}
  \item HTTP/FTP/MQTT server on top of plain sockets.
  \item Sudoku
    \begin{itemize}
      \descitem{Generator} Generate puzzles under consideration of difficulty, symmetry and ensuring a single solution.
      \descitem{Persistence} Marshall\idx{Marshalling} the datastructure and store it to a file. Then read and unmarchall it again.
      \descitem{Pretty-Printing} Code that can \textsl{pretty-print}\idx{Pretty-printing} either using ASCII art or to PDF.
      \descitem{Solver}\idx{Solver} Code that can solve a puzzle.
    \end{itemize}
  \item Sieve of Eratosthenes\footnote{\url{https://en.wikipedia.org/wiki/Sieve_of_Eratosthenes}}
\end{itemize}

\section{Make}

Make\idx{Make} is a general-purpose build system where build target and subtargets can be defined along with dependencies and recipes (sequences of commands) for building them. The general-purpose comes from it being able to run arbitrary commands and thus support build flows where parts of the codebase is autogenerated. For instance, it is a small matter to depine a build target with dependencies for each of the following steps:
\begin{itemize}
  \item Build some rust code for doing graph analysis.
  \item Run a python script that loads a graph from JSON, uses the rust code to produce some analysis, and dump the result into some C file.
  \item Build a C program that depends on that C file.
  \item Run that C program to produce an image.
  \item Run a command that generates a PDF from that image.
\end{itemize}

% dependency resolution
By analysing the dependency\idxx{Graph!Dependency}{Dependency graph}\index{Dependency graph} graph in the context of the timestamps of already built targets, it performs dependency\idx{Dependency resolution} resolution (i.e., comes up with a minimal build plan) and executes it. Make is by no means the only system that can do this. However, it is by far the simplest one to get started with.

\subsection{Variables}

\subsection{Targets}

\subsection{Patterns}

\section{C}

\idx{C}Low-level language that adds syntax for the fundamental\idx{Fundamental abstractions} abstractions in data and logic to assembly. This makes it a simple language to learn, and just as simple a language to mess up in. It has a precompiler that is very convenient for producing software product lines\idx{Product line}.

\subsection{Precompiler}
\subsection{Pointer Arithmetic}


{
\setmonofont[
  Contextuals={Alternate}
]{Fira Code}

\section{Elixir}

\idx{Elixir}A distributed language that is built around the actor model with a focus on concurrency\idx{Concurrency} and availability\idx{Availability}. It is a modern language with a long list of attractive features. Its main downside is that it is not a numerically fast language.

% beam

% who uses beam
BEAM is not just a virtual machine for yet another group of small languages. BEAM (e.g., Erlang and Elixir) has a few users, including: 
\begin{itemize}
  \descitem{Discord} is using Elixir to scale to 5M concurrent users.
%        \\ \url{https://discord.com/blog/how-discord-scaled-elixir-to-5-000-000-concurrent-users}
  \descitem{Facebook} uses Erlang to power its chat service serving 100M active users.
%        \\ \url{https://kyan.com/news/an-introduction-to-the-elixir-programming-language}
  \descitem{WhatsApp} uses Erlang to run messaging servers, each covering 2M users.
  \descitem{Ericsson} uses Erlang for its GPRS, 3G, 4G and 5G infrastructure, and has a market share of 40\%.
  \item In 2018 \textbf{Cisco} shipped 2M devices running Erlang. At that point 90\% of all internet traffic went through Erlang based nodes.
  \item 100+ NEPs (network equipment providers) -- including the top 8 -- use Erlang based components in their products. Same numbers for SPs (service providers).
%        \\ \url{https://codesync.global/media/https-youtu-be-077-xjv6plq/}
\end{itemize}

\subsection{Execution}
\subsubsection{Compilation}
\subsubsection{Interpretation}

\begin{minted}[breaklines,fontsize=\tiny]{iex}
$ iex
Erlang/OTP 24 [erts-12.2.1] [source] [64-bit] [smp:8:8] [ds:8:8:10] [async-threads:1] [jit]

Interactive Elixir (1.12.2) - press Ctrl+C to exit (type h() ENTER for help)
iex(1)> 1+1
2
iex(2)> incr = fn i -> i + 1 end
#Function<44.65746770/1 in :erl_eval.expr/5>
iex(3)> incr.(1)
2
\end{minted}

\subsubsection{LiveBook}

\idx{Livebook}
\begin{itemize}
  \item Get LiveBook: \url{https://livebook.dev}
  \item Watch demos: \url{https://www.youtube.com/@livebookdev}
\end{itemize}

\subsection{Functions}

\subsection{Modules}

\inputminted[fontsize=\normalsize]{elixir}{../src/elixir/fibonacci.ex}

\begin{minted}[breaklines]{iex}
iex(1)> Enum.map(0..10, fn n -> Fibonacci.fib(n) end)
[0, 1, 1, 2, 3, 5, 8, 13, 21, 34, 55]
\end{minted}

\subsection{Pattern Matching}

\subsection{Pipe Operator}

\subsection{GenServer}

\subsection{Registry}

\subsection{Cache}

\subsection{Supervision}
\subsubsection{Supervisor}
\subsubsection{DynamicSupervisor}

\subsection{GenStage}
\subsubsection{Flow}

\idx{Flow}\url{https://github.com/dashbitco/flow/}

\subsubsection{Broadway}

\idx{Broadway}\url{https://elixir-broadway.org}

\subsection{Domains}

\subsubsection{Web}

% phoenix
\idx{Phoenix}\url{https://www.phoenixframework.org}

% channels

% liveviews

\subsubsection{Data Science}

\url{https://www.youtube.com/@titantech2271/videos}

\subsubsection{Internet of Things}

\idx{Nerves}\url{https://nerves-project.org}

}

\section{Go}

\idx{Go}\index{Go-lang} Go is a fast language focusing on concurrency\idx{Concurrency} under a single operating system (i.e., vertical scalability\idxx{Scalability!Vertical}{Vertical scalability}). The creators are opinionated, and you are unlike to agree on all of their opinions. But there's always the highway. Go programmers are known as \textsl{gophers}.

% link to youtube video
Rob Pike's\idxx{Pike, Rob}{Rob Pike} \quoted{Concurrency is not Parallelism} presentation is a good teaser for the language\footnote{While his examples are in Go, the same principles apply to Elixir (see section \ref{lang:elixir}).}: \url{https://www.youtube.com/watch?v=oV9rvDllKEg}

\subsection{Go Routines}
\subsection{Channels}


\section{Python}

\idx{Python}Script\idx{Script Language} language which wont bite too much. It won't do concurrency\footnote{Unless you \textsl{\textbf{really}} know what you are doing, and 95\% of those who think they do don't.}, syntax errors are produced at runtime, and it is numerically pretty slow. If you can accept this (e.g., if you are doing script work) it's a pretty good language.

\subsection{Embedding}

\subsection{Symbolic Mathematics}

\idx{SymPy}\url{https://www.sympy.org}

\section{Rust}

\idx{Rust}Rust is a modern C-replacement language that focuses on safety (both memory and thread) and speed. Rust programmers are known as \textsl{rustaceans}\idx{Rustacean}.

\subsection{Drop-In Replacement for C}
\subsection{Zero-Cost Abstraction}
\subsection{Ownership Model}


\section{Other Languages}

\begin{itemize}
  \descitem{Carbon}
  \descitem{Julia}
  \descitem{Mojo}
  \descitem{R}
\end{itemize}


