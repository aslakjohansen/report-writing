


\section{Information Models}

\idx{Information model}Structuring or modeling of information is ways that allows for flexible and efficient queries.

\subsection{Schemas}

\subsection{Models}

% text

% fig: two match sites, context and information nodes, unit, stream-id
\begin{figure}[tbp]
  \begin{center}
    \begin{tikzpicture}[remember picture]
      \newcommand{\pointsize}[0]{12mm}
      \newcommand{\datadist}[0]{10mm}
      
      \tikzstyle{dedge} = [thick,->,>=stealth,draw=black]
      \tikzstyle{node}=[
        overlay,
        circle,
        thick,
        anchor=center,
      ]
      \tikzstyle{point}=[
        node,
        draw=blue,
        minimum height=\pointsize,
        minimum width=\pointsize,
      ]
      \tikzstyle{room}=[
        node,
        draw=blue,
        minimum height=\pointsize,
        minimum width=\pointsize,
      ]
      \tikzstyle{unit}=[
        node,
        draw=purple,
        anchor=north west,
      ]
      \tikzstyle{data}=[
        node,
        draw=teal,
        anchor=north east,
      ]
      \tikzstyle{field}=[
        data,
        rectangle,
        anchor=center,
      ]
      
      \newcommand{\data}[4]{
        \node[data] (#1) at (#2) {data};
        \node[field] (hist) at ([xshift=-\datadist, yshift=-\datadist] #1.center) {#3};
        \node[field] (live) at ([xshift=\datadist, yshift=-\datadist] #1.center) {#4};
        \draw[dedge] (#1) -- (hist) node[midway,left,yshift=2mm] {hist};
        \draw[dedge] (#1) -- (live) node[midway,right,yshift=2mm] {live};
      }
      \renewcommand{\unit}[3]{
        \node[unit] (#1) at (#2) {\si{#3}};
      }
      \newcommand{\point}[5]{
        \node[point] (point) at (#1) {#2};
        \data{data}{[yshift=-4mm, xshift=-4mm-\pointsize/2] point.south}{#4}{#5};
        \unit{unit}{[yshift=-4mm, xshift=4mm+\pointsize/2] point.south}{#3};
        \draw[dedge] (point) -- (unit) node[midway,right,yshift=2mm] {unit};
        \draw[dedge] (point) -- (data) node[midway,left,yshift=2mm] {data};
      }
      
      \point{0,0}{Temp}{\degreeCelsius}{12}{17c5ae15}
      \point{5,0}{Hum}{\percent}{13}{65d0be73}
      \point{0,4}{PIR}{}{24}{5a3573c3}
      
      \node[room] (room1) at (5,4) {Room};
      \node[field] (room1area) at ([yshift=\datadist+4mm] room1.center) {17 \si{\square\meter}};
      \draw[dedge] (room1) -- (room1area) node[midway,right,yshift=-0.5mm] {area};
    \end{tikzpicture}
  \end{center}
  \caption{Example of an abstrat information model.}
  \label{fig:topics:info:model}
\end{figure}

\subsection{Patterns}

% text

% fig

\subsection{Match Sites}

% text

% fig

\subsection{Result Sets}

% text

% fig

\subsection{Ontologies}
\subsubsection{RDF}
\subsubsection{OWL}
\subsubsection{Select Ontologies}

% brick
\idx{Brick}\url{https://brickschema.org}

% web of things

% schema.org

% QUDT
The QUDT\idx{QUDT}\footnote{\url{https://www.qudt.org}} ontology models units. While it is the de-facto ontology for this, it is not particularly convenient for automatic conversion.

\subsection{Property Graphs}
\subsubsection{Neo4J}

\idx{Neo4j}\url{https://neo4j.com}

\subsection{Query Languages}
\subsubsection{Cypher}
\subsubsection{OpenCypher}

\idx{OpenCypher}\url{https://opencypher.org}

\subsubsection{Gremlin}
\subsubsection{SparQL}


