\section{Information Models}

\idx{Information model}Structuring or modeling of information is ways that allows for flexible and efficient queries.

\subsection{Schemas}

\subsection{Models}

% text
The model\idx{Model} itself is what describes that concrete domain instance. Continuing in the spirit of the building domain, figure \ref{fig:topics:info:model} illustrates what a partial model of a building could look like. This particular model contaings two rooms on the 3rd floor of some building. Each room is annotated with an area as well as three sensors covering the modalities of temperateure, relative humidity and occupancy. A PIR sensor measures the occupancy, and a humidity sensor measures the relative humidity. In one room a thermistor is measuring the temperature and in the other a thermocouple does the same. Each sensor is associated with a unit as well as a pair of data references; one for retreaving historical data (using a timeseries ID) and one for retreaving live data (using a stream ID).

% fig: two match sites, context and information nodes, unit, stream-id
\begin{figure}[tbp]
  \begin{center}
  \rotatebox{90}{
    \begin{tikzpicture}[remember picture]
      \newcommand{\nodetext}[1]{\scalebox{0.7}{#1}}
\newcommand{\hlnodetext}[1]{\nodetext{\textbf{#1}}}
\newcommand{\modalitytext}[1]{\scalebox{0.65}{#1}}
\newcommand{\hlmodalitytext}[1]{\modalitytext{\textbf{#1}}}
\newcommand{\fieldtext}[1]{\scalebox{0.7}{#1}}
\newcommand{\hlfieldtext}[1]{\fieldtext{\textbf{#1}}}
\newcommand{\edgetext}[1]{\scalebox{0.7}{\textit{#1}}}
\newcommand{\hledgetext}[1]{\edgetext{\textbf{#1}}}
\newcommand{\centerspacing}[0]{2.0}
\newcommand{\pointsize}[0]{(8mm)}
\newcommand{\datadist}[0]{(11mm)}
\newcommand{\pointdist}[0]{(\pointsize/2+9mm)}
\newcommand{\roomdist}[0]{(49mm)}
\newcommand{\sqrttwo}[0]{(1.41421356237)}
\newcommand{\diagside}[0]{(0.70710678119)}

\tikzstyle{dedge} = [->,>=stealth,draw=black]
\tikzstyle{dbedge} = [<->,>=stealth,draw=black]
\tikzstyle{node}=[
  overlay,
  circle,
  align=center,
  anchor=center,
]
\tikzstyle{point}=[
  node,
  draw=blue,
  minimum height=\pointsize,
  minimum width=\pointsize,
]
\tikzstyle{room}=[
  node,
  draw=blue,
  minimum height=\pointsize,
  minimum width=\pointsize,
]
\tikzstyle{modality}=[
  node,
  draw=purple,
]
\tikzstyle{unit}=[
  node,
  draw=teal,
  anchor=center,
]
\tikzstyle{data}=[
  node,
  draw=teal,
  anchor=center,
]
\tikzstyle{field}=[
  data,
  rectangle,
  anchor=center,
]
\tikzstyle{highlight}=[
  very thick,
]
\tikzstyle{highlighted-point}=[
  point,
  highlight,
  fill=blue!10,
]
\tikzstyle{highlighted-room}=[
  room,
  highlight,
  fill=blue!10,
]
\tikzstyle{highlighted-modality}=[
  modality,
  highlight,
  fill=purple!10,
]
\tikzstyle{highlighted-unit}=[
  unit,
  highlight,
  fill=teal!10,
]
\tikzstyle{highlighted-data}=[
  data,
  highlight,
  fill=teal!10,
]
\tikzstyle{highlighted-field}=[
  field,
  highlight,
  fill=teal!10,
]

\newcommand{\data}[6]{
  \IfSubStr{#6}{highlight:data}{
    \node[highlighted-data] (#2) at (#3) {\hlnodetext{data}};
    \node[highlighted-field] (hist) at ([yshift=-\datadist,xshift=-1*#1*\datadist] #2.center) {\hlfieldtext{#4}};
    \node[highlighted-field] (live) at ([yshift=-\datadist,xshift= 1*#1*\datadist] #2.center) {\hlfieldtext{#5}};
    \draw[dedge,highlight] (#2) -- (hist) node[midway,sloped,above] {\hledgetext{hist}};
    \draw[dedge,highlight] (#2) -- (live) node[midway,sloped,above] {\hledgetext{live}};
  }{
    \node[data] (#2) at (#3) {\nodetext{data}};
    \node[field] (hist) at ([yshift=-\datadist,xshift=-1*#1*\datadist] #2.center) {\fieldtext{#4}};
    \node[field] (live) at ([yshift=-\datadist,xshift= 1*#1*\datadist] #2.center) {\fieldtext{#5}};
    \draw[dedge] (#2) -- (hist) node[midway,sloped,above] {\edgetext{hist}};
    \draw[dedge] (#2) -- (live) node[midway,sloped,above] {\edgetext{live}};
  }
}
\renewcommand{\unit}[4]{
  \IfSubStr{#4}{highlight:unit}{
    \node[highlighted-unit] (#1) at (#2) {\hlnodetext{\si{#3}}};
  }{
    \node[unit] (#1) at (#2) {\nodetext{\si{#3}}};
  }
}
\newcommand{\point}[8][1]{
  \IfSubStr{#8}{highlight:point}{
    \node[highlighted-point] (#2 point) at (#3) {\hlnodetext{#4}};
  }{
    \node[point] (#2 point) at (#3) {\nodetext{#4}};
  }
  \data{#1}{data}{[yshift=-\pointdist,xshift=-1*#1*\pointdist] #2 point.center}{#6}{#7}{#8};
  \unit{unit}{    [yshift=-\pointdist,xshift=   #1*\pointdist] #2 point.center}{#5}{#8};
  \IfSubStr{#8}{highlight:unit}{
    \draw[dedge,highlight] (#2 point) -- (unit) node[midway,sloped,above] {\hledgetext{unit}};
  }{
    \draw[dedge] (#2 point) -- (unit) node[midway,sloped,above] {\edgetext{unit}};
  }
  \IfSubStr{#8}{highlight:data}{
    \draw[dedge,highlight] (#2 point) -- (data) node[midway,sloped,above] {\hledgetext{data}};
  }{
    \draw[dedge] (#2 point) -- (data) node[midway,sloped,above] {\edgetext{data}};
  }
}

\newcommand{\genRoomI}[3]{
  \node[room] (room1) at (#1, #2) {\nodetext{Room}};
  \node[field] (room1area) at ([yshift=\datadist+5mm] room1.center) {\fieldtext{17 \si{\square\meter}}};
  \draw[dedge] (room1) -- (room1area) node[midway,sloped,above] {\edgetext{area}};
  
  \point{room1t}{[xshift=-\roomdist*\diagside,yshift=-\roomdist*\diagside] room1}{TC}{\degreeCelsius}{12}{17c5ae15}{#3}
  \point{room1h}{[xshift=0,yshift=-\roomdist] room1}{Hum}{\percent}{13}{65d0be73}{}
  \point{room1o}{[xshift=-\roomdist,yshift=0] room1}{PIR}{}{24}{5a3573c3}{}
  
  \node[modality] (room1tempModality) at ($(room1)!0.5!(room1t point)$) {\modalitytext{temperature}};
  \draw[dedge] (room1) -- (room1tempModality) node[midway,sloped,above] {\edgetext{modality}};
  \draw[dedge] (room1t point) -- (room1tempModality) node[midway,sloped,above] {\edgetext{provides}};
  
  \node[modality] (room1humModality) at ($(room1)!0.5!(room1h point)$) {\modalitytext{relative}\\\modalitytext{humidity}};
  \draw[dedge] (room1) -- (room1humModality) node[midway,sloped,above,rotate=180] {\edgetext{modality}};
  \draw[dedge] (room1h point) -- (room1humModality) node[midway,sloped,above] {\edgetext{provides}};
  
  \node[modality] (room1occModality) at ($(room1)!0.5!(room1o point)$) {\modalitytext{occupancy}};
  \draw[dedge] (room1) -- (room1occModality) node[midway,sloped,above] {\edgetext{modality}};
  \draw[dedge] (room1o point) -- (room1occModality) node[midway,sloped,above] {\edgetext{provides}};
}

\newcommand{\genRoomII}[3]{
  \node[room] (room2) at (#1, #2) {\nodetext{Room}};
  \node[field] (room2area) at ([yshift=\datadist+5mm] room2.center) {\fieldtext{23 \si{\square\meter}}};
  \draw[dedge] (room2) -- (room2area) node[midway,sloped,above] {\edgetext{area}};
  
  \point[-1]{room2t}{[xshift=\roomdist*\diagside,yshift=-\roomdist*\diagside] room2}{TM}{\degreeCelsius}{16}{3b83ce7c}{}
  \point[-1]{room2h}{[xshift=0,yshift=-\roomdist] room2}{Hum}{\percent}{53}{91729592}{}
  \point[-1]{room2o}{[xshift=\roomdist,yshift=0] room2}{PIR}{}{27}{b4ae871d}{}
  
  \node[modality] (room2tempModality) at ($(room2)!0.5!(room2t point)$) {\modalitytext{temperature}};
  \draw[dedge] (room2) -- (room2tempModality) node[midway,sloped,above] {\edgetext{modality}};
  \draw[dedge] (room2t point) -- (room2tempModality) node[midway,sloped,above] {\edgetext{provides}};
  
  \node[modality] (room2humModality) at ($(room2)!0.5!(room2h point)$) {\modalitytext{relative}\\\modalitytext{humidity}};
  \draw[dedge] (room2) -- (room2humModality) node[midway,sloped,above,rotate=180] {\edgetext{modality}};
  \draw[dedge] (room2h point) -- (room2humModality) node[midway,sloped,above] {\edgetext{provides}};
  
  \node[modality] (room2occModality) at ($(room2)!0.5!(room2o point)$) {\modalitytext{occupancy}};
  \draw[dedge] (room2) -- (room2occModality) node[midway,sloped,above] {\edgetext{modality}};
  \draw[dedge] (room2o point) -- (room2occModality) node[midway,sloped,above] {\edgetext{provides}};
}

\newcommand{\genFloor}[0]{
  \draw[dbedge] (room1) -- (room2) node[midway,sloped,above] {\edgetext{adjacent}};
  
  \node[room] (floor) at ([yshift=14mm] $(room1)!0.5!(room2)$) {\nodetext{Floor}};
  \draw[dedge] (floor) -- (room1) node[midway,sloped,above] {\edgetext{contains}};
  \draw[dedge] (floor) -- (room2) node[midway,sloped,above] {\edgetext{contains}};
  
  \node[field] (floorName) at ([yshift=\datadist+6mm] floor.center) {\fieldtext{3rd}};
  \draw[dedge] (floor) -- (floorName) node[midway,sloped,above] {\edgetext{name}};
}



      
      \genRoomI{-\centerspacing}{0}{}
      \genRoomII{\centerspacing}{0}{}
      \genFloor
    \end{tikzpicture}
  }
  \end{center}
  \caption[Example of an abstract information model]{Example of an abstract information model covering two rooms on a single floor and some static and dynamic data associated with these. TM is an abbreviation for thermistor, and TC for thermocouple.}
  \label{fig:topics:info:model}
\end{figure}

\subsection{Query Patterns}

% problem: extracting information from a model, example application (want to list the absolute humidity of all room organized by floor), what we need (temp data, rel hum data, floor name)

% solution: query execution engine, traversing enitre model for sites where the pattern matches, what is matched? what is captured?

% fig: pattern covering [room, room area, occupancy provider, temperature provider, relhum provider]
\begin{figure}[tbp]
  \begin{center}
    \begin{tikzpicture}[remember picture]
      \newcommand{\nodetext}[1]{\scalebox{0.7}{#1}}
      \newcommand{\modalitytext}[1]{\scalebox{0.65}{#1}}
      \newcommand{\fieldtext}[1]{\scalebox{0.7}{#1}}
      \newcommand{\edgetext}[1]{\scalebox{0.7}{\textit{#1}}}
      
      \newcommand{\pointsize}[0]{(8mm)}
      \newcommand{\dist}[0]{(24mm)}
      
      \tikzstyle{dedge} = [->,>=stealth,draw=black]
      \tikzstyle{dbedge} = [<->,>=stealth,draw=black]
      \tikzstyle{node}=[
        overlay,
        circle,
        align=center,
        anchor=center,
      ]
      \tikzstyle{point}=[
        node,
        draw=blue,
        minimum height=\pointsize,
        minimum width=\pointsize,
      ]
      \tikzstyle{room}=[
        node,
        draw=blue,
        minimum height=\pointsize,
        minimum width=\pointsize,
      ]
      \tikzstyle{modality}=[
        node,
        draw=purple,
      ]
      \tikzstyle{unit}=[
        node,
        draw=teal,
        anchor=center,
      ]
      \tikzstyle{data}=[
        node,
        draw=teal,
        anchor=center,
      ]
      \tikzstyle{field}=[
        data,
        rectangle,
        anchor=center,
      ]
      
      \node[room] (room) at (0,0) {Room};
      \node[room] (floor) at ([xshift=30mm] room.center) {Floor};
      \node[data] (floorname) at ([xshift=22mm] floor.center) {?};
      \draw[dedge] (floor) -- (room) node[midway,sloped,above] {contains};
      \draw[dedge] (floor) -- (floorname) node[midway,sloped,above] {name};
      
      \node[modality] (tmodality) at ([xshift=-26mm, yshift=-26mm] room.center) {temperature};
      \node[modality] (hmodality) at ([xshift=-26mm, yshift= 26mm] room.center) {relative\\humidity};
      \node[room] (tsensor) at ([xshift=-32mm] tmodality.center) {?};
      \node[room] (hsensor) at ([xshift=-32mm] hmodality.center) {?};
      \draw[dedge] (room) -- (tmodality) node[midway,sloped,above] {modality};
      \draw[dedge] (room) -- (hmodality) node[midway,sloped,above] {modality};
      \draw[dedge] (tsensor) -- (tmodality) node[midway,sloped,above] {provides};
      \draw[dedge] (hsensor) -- (hmodality) node[midway,sloped,above] {provides};
    \end{tikzpicture}
  \end{center}
  \vspace{8mm}
  \caption[Example of the pattern of an abstract model query]{Example of the pattern of an abstract model query. It matches combinations of temperature and relative humidity sensors that are associated with a specific room, along with the floor name associated with the floor of that room.}
  \label{fig:topics:info:query:pattern}
\end{figure}

\subsection{Match Sites}

% text

% fig: scale vbox of [match site 1, match site 2, all unit matches, all data matches]
\begin{figure}[tbp]
  \centering
  \begin{subfigure}[b]{0.48\textwidth}
    \rotatebox{90}{
      \scalebox{0.5}{
        \begin{tikzpicture}[remember picture]
          \newcommand{\nodetext}[1]{\scalebox{0.7}{#1}}
\newcommand{\hlnodetext}[1]{\nodetext{\textbf{#1}}}
\newcommand{\modalitytext}[1]{\scalebox{0.65}{#1}}
\newcommand{\hlmodalitytext}[1]{\modalitytext{\textbf{#1}}}
\newcommand{\fieldtext}[1]{\scalebox{0.7}{#1}}
\newcommand{\hlfieldtext}[1]{\fieldtext{\textbf{#1}}}
\newcommand{\edgetext}[1]{\scalebox{0.7}{\textit{#1}}}
\newcommand{\hledgetext}[1]{\edgetext{\textbf{#1}}}
\newcommand{\centerspacing}[0]{2.0}
\newcommand{\pointsize}[0]{(8mm)}
\newcommand{\datadist}[0]{(11mm)}
\newcommand{\pointdist}[0]{(\pointsize/2+9mm)}
\newcommand{\roomdist}[0]{(49mm)}
\newcommand{\sqrttwo}[0]{(1.41421356237)}
\newcommand{\diagside}[0]{(0.70710678119)}

\tikzstyle{dedge} = [->,>=stealth,draw=black]
\tikzstyle{dbedge} = [<->,>=stealth,draw=black]
\tikzstyle{node}=[
  overlay,
  circle,
  align=center,
  anchor=center,
]
\tikzstyle{point}=[
  node,
  draw=blue,
  minimum height=\pointsize,
  minimum width=\pointsize,
]
\tikzstyle{room}=[
  node,
  draw=blue,
  minimum height=\pointsize,
  minimum width=\pointsize,
]
\tikzstyle{modality}=[
  node,
  draw=purple,
]
\tikzstyle{unit}=[
  node,
  draw=teal,
  anchor=center,
]
\tikzstyle{data}=[
  node,
  draw=teal,
  anchor=center,
]
\tikzstyle{field}=[
  data,
  rectangle,
  anchor=center,
]
\tikzstyle{highlight}=[
  very thick,
]
\tikzstyle{highlighted-point}=[
  point,
  highlight,
  fill=blue!10,
]
\tikzstyle{highlighted-room}=[
  room,
  highlight,
  fill=blue!10,
]
\tikzstyle{highlighted-modality}=[
  modality,
  highlight,
  fill=purple!10,
]
\tikzstyle{highlighted-unit}=[
  unit,
  highlight,
  fill=teal!10,
]
\tikzstyle{highlighted-data}=[
  data,
  highlight,
  fill=teal!10,
]
\tikzstyle{highlighted-field}=[
  field,
  highlight,
  fill=teal!10,
]

\newcommand{\data}[6]{
  \IfSubStr{#6}{highlight:data}{
    \node[highlighted-data] (#2) at (#3) {\hlnodetext{data}};
    \node[highlighted-field] (hist) at ([yshift=-\datadist,xshift=-1*#1*\datadist] #2.center) {\hlfieldtext{#4}};
    \node[highlighted-field] (live) at ([yshift=-\datadist,xshift= 1*#1*\datadist] #2.center) {\hlfieldtext{#5}};
    \draw[dedge,highlight] (#2) -- (hist) node[midway,sloped,above] {\hledgetext{hist}};
    \draw[dedge,highlight] (#2) -- (live) node[midway,sloped,above] {\hledgetext{live}};
  }{
    \node[data] (#2) at (#3) {\nodetext{data}};
    \node[field] (hist) at ([yshift=-\datadist,xshift=-1*#1*\datadist] #2.center) {\fieldtext{#4}};
    \node[field] (live) at ([yshift=-\datadist,xshift= 1*#1*\datadist] #2.center) {\fieldtext{#5}};
    \draw[dedge] (#2) -- (hist) node[midway,sloped,above] {\edgetext{hist}};
    \draw[dedge] (#2) -- (live) node[midway,sloped,above] {\edgetext{live}};
  }
}
\renewcommand{\unit}[4]{
  \IfSubStr{#4}{highlight:unit}{
    \node[highlighted-unit] (#1) at (#2) {\hlnodetext{\si{#3}}};
  }{
    \node[unit] (#1) at (#2) {\nodetext{\si{#3}}};
  }
}
\newcommand{\point}[8][1]{
  \IfSubStr{#8}{highlight:point}{
    \node[highlighted-point] (#2 point) at (#3) {\hlnodetext{#4}};
  }{
    \node[point] (#2 point) at (#3) {\nodetext{#4}};
  }
  \data{#1}{data}{[yshift=-\pointdist,xshift=-1*#1*\pointdist] #2 point.center}{#6}{#7}{#8};
  \unit{unit}{    [yshift=-\pointdist,xshift=   #1*\pointdist] #2 point.center}{#5}{#8};
  \IfSubStr{#8}{highlight:unit}{
    \draw[dedge,highlight] (#2 point) -- (unit) node[midway,sloped,above] {\hledgetext{unit}};
  }{
    \draw[dedge] (#2 point) -- (unit) node[midway,sloped,above] {\edgetext{unit}};
  }
  \IfSubStr{#8}{highlight:data}{
    \draw[dedge,highlight] (#2 point) -- (data) node[midway,sloped,above] {\hledgetext{data}};
  }{
    \draw[dedge] (#2 point) -- (data) node[midway,sloped,above] {\edgetext{data}};
  }
}

\newcommand{\genRoomI}[3]{
  \node[room] (room1) at (#1, #2) {\nodetext{Room}};
  \node[field] (room1area) at ([yshift=\datadist+5mm] room1.center) {\fieldtext{17 \si{\square\meter}}};
  \draw[dedge] (room1) -- (room1area) node[midway,sloped,above] {\edgetext{area}};
  
  \point{room1t}{[xshift=-\roomdist*\diagside,yshift=-\roomdist*\diagside] room1}{TC}{\degreeCelsius}{12}{17c5ae15}{#3}
  \point{room1h}{[xshift=0,yshift=-\roomdist] room1}{Hum}{\percent}{13}{65d0be73}{}
  \point{room1o}{[xshift=-\roomdist,yshift=0] room1}{PIR}{}{24}{5a3573c3}{}
  
  \node[modality] (room1tempModality) at ($(room1)!0.5!(room1t point)$) {\modalitytext{temperature}};
  \draw[dedge] (room1) -- (room1tempModality) node[midway,sloped,above] {\edgetext{modality}};
  \draw[dedge] (room1t point) -- (room1tempModality) node[midway,sloped,above] {\edgetext{provides}};
  
  \node[modality] (room1humModality) at ($(room1)!0.5!(room1h point)$) {\modalitytext{relative}\\\modalitytext{humidity}};
  \draw[dedge] (room1) -- (room1humModality) node[midway,sloped,above,rotate=180] {\edgetext{modality}};
  \draw[dedge] (room1h point) -- (room1humModality) node[midway,sloped,above] {\edgetext{provides}};
  
  \node[modality] (room1occModality) at ($(room1)!0.5!(room1o point)$) {\modalitytext{occupancy}};
  \draw[dedge] (room1) -- (room1occModality) node[midway,sloped,above] {\edgetext{modality}};
  \draw[dedge] (room1o point) -- (room1occModality) node[midway,sloped,above] {\edgetext{provides}};
}

\newcommand{\genRoomII}[3]{
  \node[room] (room2) at (#1, #2) {\nodetext{Room}};
  \node[field] (room2area) at ([yshift=\datadist+5mm] room2.center) {\fieldtext{23 \si{\square\meter}}};
  \draw[dedge] (room2) -- (room2area) node[midway,sloped,above] {\edgetext{area}};
  
  \point[-1]{room2t}{[xshift=\roomdist*\diagside,yshift=-\roomdist*\diagside] room2}{TM}{\degreeCelsius}{16}{3b83ce7c}{}
  \point[-1]{room2h}{[xshift=0,yshift=-\roomdist] room2}{Hum}{\percent}{53}{91729592}{}
  \point[-1]{room2o}{[xshift=\roomdist,yshift=0] room2}{PIR}{}{27}{b4ae871d}{}
  
  \node[modality] (room2tempModality) at ($(room2)!0.5!(room2t point)$) {\modalitytext{temperature}};
  \draw[dedge] (room2) -- (room2tempModality) node[midway,sloped,above] {\edgetext{modality}};
  \draw[dedge] (room2t point) -- (room2tempModality) node[midway,sloped,above] {\edgetext{provides}};
  
  \node[modality] (room2humModality) at ($(room2)!0.5!(room2h point)$) {\modalitytext{relative}\\\modalitytext{humidity}};
  \draw[dedge] (room2) -- (room2humModality) node[midway,sloped,above,rotate=180] {\edgetext{modality}};
  \draw[dedge] (room2h point) -- (room2humModality) node[midway,sloped,above] {\edgetext{provides}};
  
  \node[modality] (room2occModality) at ($(room2)!0.5!(room2o point)$) {\modalitytext{occupancy}};
  \draw[dedge] (room2) -- (room2occModality) node[midway,sloped,above] {\edgetext{modality}};
  \draw[dedge] (room2o point) -- (room2occModality) node[midway,sloped,above] {\edgetext{provides}};
}

\newcommand{\genFloor}[0]{
  \draw[dbedge] (room1) -- (room2) node[midway,sloped,above] {\edgetext{adjacent}};
  
  \node[room] (floor) at ([yshift=14mm] $(room1)!0.5!(room2)$) {\nodetext{Floor}};
  \draw[dedge] (floor) -- (room1) node[midway,sloped,above] {\edgetext{contains}};
  \draw[dedge] (floor) -- (room2) node[midway,sloped,above] {\edgetext{contains}};
  
  \node[field] (floorName) at ([yshift=\datadist+6mm] floor.center) {\fieldtext{3rd}};
  \draw[dedge] (floor) -- (floorName) node[midway,sloped,above] {\edgetext{name}};
}



          
          \genRoomI{-\centerspacing}{0}{highlight:temp,highlight:hum}
          \genRoomII{\centerspacing}{0}{}
          \genFloor
        \end{tikzpicture}
      }
    }
    \caption{1st match site of absolute humidity query.}
    \label{fig:topics:info:match:absI}
  \end{subfigure}
  \hfill
  \begin{subfigure}[b]{0.48\textwidth}
    \rotatebox{90}{
      \scalebox{0.5}{
        \begin{tikzpicture}[remember picture]
          \newcommand{\nodetext}[1]{\scalebox{0.7}{#1}}
\newcommand{\hlnodetext}[1]{\nodetext{\textbf{#1}}}
\newcommand{\modalitytext}[1]{\scalebox{0.65}{#1}}
\newcommand{\hlmodalitytext}[1]{\modalitytext{\textbf{#1}}}
\newcommand{\fieldtext}[1]{\scalebox{0.7}{#1}}
\newcommand{\hlfieldtext}[1]{\fieldtext{\textbf{#1}}}
\newcommand{\edgetext}[1]{\scalebox{0.7}{\textit{#1}}}
\newcommand{\hledgetext}[1]{\edgetext{\textbf{#1}}}
\newcommand{\centerspacing}[0]{2.0}
\newcommand{\pointsize}[0]{(8mm)}
\newcommand{\datadist}[0]{(11mm)}
\newcommand{\pointdist}[0]{(\pointsize/2+9mm)}
\newcommand{\roomdist}[0]{(49mm)}
\newcommand{\sqrttwo}[0]{(1.41421356237)}
\newcommand{\diagside}[0]{(0.70710678119)}

\tikzstyle{dedge} = [->,>=stealth,draw=black]
\tikzstyle{dbedge} = [<->,>=stealth,draw=black]
\tikzstyle{node}=[
  overlay,
  circle,
  align=center,
  anchor=center,
]
\tikzstyle{point}=[
  node,
  draw=blue,
  minimum height=\pointsize,
  minimum width=\pointsize,
]
\tikzstyle{room}=[
  node,
  draw=blue,
  minimum height=\pointsize,
  minimum width=\pointsize,
]
\tikzstyle{modality}=[
  node,
  draw=purple,
]
\tikzstyle{unit}=[
  node,
  draw=teal,
  anchor=center,
]
\tikzstyle{data}=[
  node,
  draw=teal,
  anchor=center,
]
\tikzstyle{field}=[
  data,
  rectangle,
  anchor=center,
]
\tikzstyle{highlight}=[
  very thick,
]
\tikzstyle{highlighted-point}=[
  point,
  highlight,
  fill=blue!10,
]
\tikzstyle{highlighted-room}=[
  room,
  highlight,
  fill=blue!10,
]
\tikzstyle{highlighted-modality}=[
  modality,
  highlight,
  fill=purple!10,
]
\tikzstyle{highlighted-unit}=[
  unit,
  highlight,
  fill=teal!10,
]
\tikzstyle{highlighted-data}=[
  data,
  highlight,
  fill=teal!10,
]
\tikzstyle{highlighted-field}=[
  field,
  highlight,
  fill=teal!10,
]

\newcommand{\data}[6]{
  \IfSubStr{#6}{highlight:data}{
    \node[highlighted-data] (#2) at (#3) {\hlnodetext{data}};
    \node[highlighted-field] (hist) at ([yshift=-\datadist,xshift=-1*#1*\datadist] #2.center) {\hlfieldtext{#4}};
    \node[highlighted-field] (live) at ([yshift=-\datadist,xshift= 1*#1*\datadist] #2.center) {\hlfieldtext{#5}};
    \draw[dedge,highlight] (#2) -- (hist) node[midway,sloped,above] {\hledgetext{hist}};
    \draw[dedge,highlight] (#2) -- (live) node[midway,sloped,above] {\hledgetext{live}};
  }{
    \node[data] (#2) at (#3) {\nodetext{data}};
    \node[field] (hist) at ([yshift=-\datadist,xshift=-1*#1*\datadist] #2.center) {\fieldtext{#4}};
    \node[field] (live) at ([yshift=-\datadist,xshift= 1*#1*\datadist] #2.center) {\fieldtext{#5}};
    \draw[dedge] (#2) -- (hist) node[midway,sloped,above] {\edgetext{hist}};
    \draw[dedge] (#2) -- (live) node[midway,sloped,above] {\edgetext{live}};
  }
}
\renewcommand{\unit}[4]{
  \IfSubStr{#4}{highlight:unit}{
    \node[highlighted-unit] (#1) at (#2) {\hlnodetext{\si{#3}}};
  }{
    \node[unit] (#1) at (#2) {\nodetext{\si{#3}}};
  }
}
\newcommand{\point}[8][1]{
  \IfSubStr{#8}{highlight:point}{
    \node[highlighted-point] (#2 point) at (#3) {\hlnodetext{#4}};
  }{
    \node[point] (#2 point) at (#3) {\nodetext{#4}};
  }
  \data{#1}{data}{[yshift=-\pointdist,xshift=-1*#1*\pointdist] #2 point.center}{#6}{#7}{#8};
  \unit{unit}{    [yshift=-\pointdist,xshift=   #1*\pointdist] #2 point.center}{#5}{#8};
  \IfSubStr{#8}{highlight:unit}{
    \draw[dedge,highlight] (#2 point) -- (unit) node[midway,sloped,above] {\hledgetext{unit}};
  }{
    \draw[dedge] (#2 point) -- (unit) node[midway,sloped,above] {\edgetext{unit}};
  }
  \IfSubStr{#8}{highlight:data}{
    \draw[dedge,highlight] (#2 point) -- (data) node[midway,sloped,above] {\hledgetext{data}};
  }{
    \draw[dedge] (#2 point) -- (data) node[midway,sloped,above] {\edgetext{data}};
  }
}

\newcommand{\genRoomI}[3]{
  \node[room] (room1) at (#1, #2) {\nodetext{Room}};
  \node[field] (room1area) at ([yshift=\datadist+5mm] room1.center) {\fieldtext{17 \si{\square\meter}}};
  \draw[dedge] (room1) -- (room1area) node[midway,sloped,above] {\edgetext{area}};
  
  \point{room1t}{[xshift=-\roomdist*\diagside,yshift=-\roomdist*\diagside] room1}{TC}{\degreeCelsius}{12}{17c5ae15}{#3}
  \point{room1h}{[xshift=0,yshift=-\roomdist] room1}{Hum}{\percent}{13}{65d0be73}{}
  \point{room1o}{[xshift=-\roomdist,yshift=0] room1}{PIR}{}{24}{5a3573c3}{}
  
  \node[modality] (room1tempModality) at ($(room1)!0.5!(room1t point)$) {\modalitytext{temperature}};
  \draw[dedge] (room1) -- (room1tempModality) node[midway,sloped,above] {\edgetext{modality}};
  \draw[dedge] (room1t point) -- (room1tempModality) node[midway,sloped,above] {\edgetext{provides}};
  
  \node[modality] (room1humModality) at ($(room1)!0.5!(room1h point)$) {\modalitytext{relative}\\\modalitytext{humidity}};
  \draw[dedge] (room1) -- (room1humModality) node[midway,sloped,above,rotate=180] {\edgetext{modality}};
  \draw[dedge] (room1h point) -- (room1humModality) node[midway,sloped,above] {\edgetext{provides}};
  
  \node[modality] (room1occModality) at ($(room1)!0.5!(room1o point)$) {\modalitytext{occupancy}};
  \draw[dedge] (room1) -- (room1occModality) node[midway,sloped,above] {\edgetext{modality}};
  \draw[dedge] (room1o point) -- (room1occModality) node[midway,sloped,above] {\edgetext{provides}};
}

\newcommand{\genRoomII}[3]{
  \node[room] (room2) at (#1, #2) {\nodetext{Room}};
  \node[field] (room2area) at ([yshift=\datadist+5mm] room2.center) {\fieldtext{23 \si{\square\meter}}};
  \draw[dedge] (room2) -- (room2area) node[midway,sloped,above] {\edgetext{area}};
  
  \point[-1]{room2t}{[xshift=\roomdist*\diagside,yshift=-\roomdist*\diagside] room2}{TM}{\degreeCelsius}{16}{3b83ce7c}{}
  \point[-1]{room2h}{[xshift=0,yshift=-\roomdist] room2}{Hum}{\percent}{53}{91729592}{}
  \point[-1]{room2o}{[xshift=\roomdist,yshift=0] room2}{PIR}{}{27}{b4ae871d}{}
  
  \node[modality] (room2tempModality) at ($(room2)!0.5!(room2t point)$) {\modalitytext{temperature}};
  \draw[dedge] (room2) -- (room2tempModality) node[midway,sloped,above] {\edgetext{modality}};
  \draw[dedge] (room2t point) -- (room2tempModality) node[midway,sloped,above] {\edgetext{provides}};
  
  \node[modality] (room2humModality) at ($(room2)!0.5!(room2h point)$) {\modalitytext{relative}\\\modalitytext{humidity}};
  \draw[dedge] (room2) -- (room2humModality) node[midway,sloped,above,rotate=180] {\edgetext{modality}};
  \draw[dedge] (room2h point) -- (room2humModality) node[midway,sloped,above] {\edgetext{provides}};
  
  \node[modality] (room2occModality) at ($(room2)!0.5!(room2o point)$) {\modalitytext{occupancy}};
  \draw[dedge] (room2) -- (room2occModality) node[midway,sloped,above] {\edgetext{modality}};
  \draw[dedge] (room2o point) -- (room2occModality) node[midway,sloped,above] {\edgetext{provides}};
}

\newcommand{\genFloor}[0]{
  \draw[dbedge] (room1) -- (room2) node[midway,sloped,above] {\edgetext{adjacent}};
  
  \node[room] (floor) at ([yshift=14mm] $(room1)!0.5!(room2)$) {\nodetext{Floor}};
  \draw[dedge] (floor) -- (room1) node[midway,sloped,above] {\edgetext{contains}};
  \draw[dedge] (floor) -- (room2) node[midway,sloped,above] {\edgetext{contains}};
  
  \node[field] (floorName) at ([yshift=\datadist+6mm] floor.center) {\fieldtext{3rd}};
  \draw[dedge] (floor) -- (floorName) node[midway,sloped,above] {\edgetext{name}};
}



          
          \genRoomI{-\centerspacing}{0}{}
          \genRoomII{\centerspacing}{0}{highlight:temp,highlight:hump}
          \genFloor
        \end{tikzpicture}
      }
    }
    \caption{2nd match site of absolute humidity query.}
    \label{fig:topics:info:match:absII}
  \end{subfigure}
  \\
  \begin{subfigure}[b]{0.48\textwidth}
    \rotatebox{90}{
      \scalebox{0.5}{
        \begin{tikzpicture}[remember picture]
          \newcommand{\nodetext}[1]{\scalebox{0.7}{#1}}
\newcommand{\hlnodetext}[1]{\nodetext{\textbf{#1}}}
\newcommand{\modalitytext}[1]{\scalebox{0.65}{#1}}
\newcommand{\hlmodalitytext}[1]{\modalitytext{\textbf{#1}}}
\newcommand{\fieldtext}[1]{\scalebox{0.7}{#1}}
\newcommand{\hlfieldtext}[1]{\fieldtext{\textbf{#1}}}
\newcommand{\edgetext}[1]{\scalebox{0.7}{\textit{#1}}}
\newcommand{\hledgetext}[1]{\edgetext{\textbf{#1}}}
\newcommand{\centerspacing}[0]{2.0}
\newcommand{\pointsize}[0]{(8mm)}
\newcommand{\datadist}[0]{(11mm)}
\newcommand{\pointdist}[0]{(\pointsize/2+9mm)}
\newcommand{\roomdist}[0]{(49mm)}
\newcommand{\sqrttwo}[0]{(1.41421356237)}
\newcommand{\diagside}[0]{(0.70710678119)}

\tikzstyle{dedge} = [->,>=stealth,draw=black]
\tikzstyle{dbedge} = [<->,>=stealth,draw=black]
\tikzstyle{node}=[
  overlay,
  circle,
  align=center,
  anchor=center,
]
\tikzstyle{point}=[
  node,
  draw=blue,
  minimum height=\pointsize,
  minimum width=\pointsize,
]
\tikzstyle{room}=[
  node,
  draw=blue,
  minimum height=\pointsize,
  minimum width=\pointsize,
]
\tikzstyle{modality}=[
  node,
  draw=purple,
]
\tikzstyle{unit}=[
  node,
  draw=teal,
  anchor=center,
]
\tikzstyle{data}=[
  node,
  draw=teal,
  anchor=center,
]
\tikzstyle{field}=[
  data,
  rectangle,
  anchor=center,
]
\tikzstyle{highlight}=[
  very thick,
]
\tikzstyle{highlighted-point}=[
  point,
  highlight,
  fill=blue!10,
]
\tikzstyle{highlighted-room}=[
  room,
  highlight,
  fill=blue!10,
]
\tikzstyle{highlighted-modality}=[
  modality,
  highlight,
  fill=purple!10,
]
\tikzstyle{highlighted-unit}=[
  unit,
  highlight,
  fill=teal!10,
]
\tikzstyle{highlighted-data}=[
  data,
  highlight,
  fill=teal!10,
]
\tikzstyle{highlighted-field}=[
  field,
  highlight,
  fill=teal!10,
]

\newcommand{\data}[6]{
  \IfSubStr{#6}{highlight:data}{
    \node[highlighted-data] (#2) at (#3) {\hlnodetext{data}};
    \node[highlighted-field] (hist) at ([yshift=-\datadist,xshift=-1*#1*\datadist] #2.center) {\hlfieldtext{#4}};
    \node[highlighted-field] (live) at ([yshift=-\datadist,xshift= 1*#1*\datadist] #2.center) {\hlfieldtext{#5}};
    \draw[dedge,highlight] (#2) -- (hist) node[midway,sloped,above] {\hledgetext{hist}};
    \draw[dedge,highlight] (#2) -- (live) node[midway,sloped,above] {\hledgetext{live}};
  }{
    \node[data] (#2) at (#3) {\nodetext{data}};
    \node[field] (hist) at ([yshift=-\datadist,xshift=-1*#1*\datadist] #2.center) {\fieldtext{#4}};
    \node[field] (live) at ([yshift=-\datadist,xshift= 1*#1*\datadist] #2.center) {\fieldtext{#5}};
    \draw[dedge] (#2) -- (hist) node[midway,sloped,above] {\edgetext{hist}};
    \draw[dedge] (#2) -- (live) node[midway,sloped,above] {\edgetext{live}};
  }
}
\renewcommand{\unit}[4]{
  \IfSubStr{#4}{highlight:unit}{
    \node[highlighted-unit] (#1) at (#2) {\hlnodetext{\si{#3}}};
  }{
    \node[unit] (#1) at (#2) {\nodetext{\si{#3}}};
  }
}
\newcommand{\point}[8][1]{
  \IfSubStr{#8}{highlight:point}{
    \node[highlighted-point] (#2 point) at (#3) {\hlnodetext{#4}};
  }{
    \node[point] (#2 point) at (#3) {\nodetext{#4}};
  }
  \data{#1}{data}{[yshift=-\pointdist,xshift=-1*#1*\pointdist] #2 point.center}{#6}{#7}{#8};
  \unit{unit}{    [yshift=-\pointdist,xshift=   #1*\pointdist] #2 point.center}{#5}{#8};
  \IfSubStr{#8}{highlight:unit}{
    \draw[dedge,highlight] (#2 point) -- (unit) node[midway,sloped,above] {\hledgetext{unit}};
  }{
    \draw[dedge] (#2 point) -- (unit) node[midway,sloped,above] {\edgetext{unit}};
  }
  \IfSubStr{#8}{highlight:data}{
    \draw[dedge,highlight] (#2 point) -- (data) node[midway,sloped,above] {\hledgetext{data}};
  }{
    \draw[dedge] (#2 point) -- (data) node[midway,sloped,above] {\edgetext{data}};
  }
}

\newcommand{\genRoomI}[3]{
  \node[room] (room1) at (#1, #2) {\nodetext{Room}};
  \node[field] (room1area) at ([yshift=\datadist+5mm] room1.center) {\fieldtext{17 \si{\square\meter}}};
  \draw[dedge] (room1) -- (room1area) node[midway,sloped,above] {\edgetext{area}};
  
  \point{room1t}{[xshift=-\roomdist*\diagside,yshift=-\roomdist*\diagside] room1}{TC}{\degreeCelsius}{12}{17c5ae15}{#3}
  \point{room1h}{[xshift=0,yshift=-\roomdist] room1}{Hum}{\percent}{13}{65d0be73}{}
  \point{room1o}{[xshift=-\roomdist,yshift=0] room1}{PIR}{}{24}{5a3573c3}{}
  
  \node[modality] (room1tempModality) at ($(room1)!0.5!(room1t point)$) {\modalitytext{temperature}};
  \draw[dedge] (room1) -- (room1tempModality) node[midway,sloped,above] {\edgetext{modality}};
  \draw[dedge] (room1t point) -- (room1tempModality) node[midway,sloped,above] {\edgetext{provides}};
  
  \node[modality] (room1humModality) at ($(room1)!0.5!(room1h point)$) {\modalitytext{relative}\\\modalitytext{humidity}};
  \draw[dedge] (room1) -- (room1humModality) node[midway,sloped,above,rotate=180] {\edgetext{modality}};
  \draw[dedge] (room1h point) -- (room1humModality) node[midway,sloped,above] {\edgetext{provides}};
  
  \node[modality] (room1occModality) at ($(room1)!0.5!(room1o point)$) {\modalitytext{occupancy}};
  \draw[dedge] (room1) -- (room1occModality) node[midway,sloped,above] {\edgetext{modality}};
  \draw[dedge] (room1o point) -- (room1occModality) node[midway,sloped,above] {\edgetext{provides}};
}

\newcommand{\genRoomII}[3]{
  \node[room] (room2) at (#1, #2) {\nodetext{Room}};
  \node[field] (room2area) at ([yshift=\datadist+5mm] room2.center) {\fieldtext{23 \si{\square\meter}}};
  \draw[dedge] (room2) -- (room2area) node[midway,sloped,above] {\edgetext{area}};
  
  \point[-1]{room2t}{[xshift=\roomdist*\diagside,yshift=-\roomdist*\diagside] room2}{TM}{\degreeCelsius}{16}{3b83ce7c}{}
  \point[-1]{room2h}{[xshift=0,yshift=-\roomdist] room2}{Hum}{\percent}{53}{91729592}{}
  \point[-1]{room2o}{[xshift=\roomdist,yshift=0] room2}{PIR}{}{27}{b4ae871d}{}
  
  \node[modality] (room2tempModality) at ($(room2)!0.5!(room2t point)$) {\modalitytext{temperature}};
  \draw[dedge] (room2) -- (room2tempModality) node[midway,sloped,above] {\edgetext{modality}};
  \draw[dedge] (room2t point) -- (room2tempModality) node[midway,sloped,above] {\edgetext{provides}};
  
  \node[modality] (room2humModality) at ($(room2)!0.5!(room2h point)$) {\modalitytext{relative}\\\modalitytext{humidity}};
  \draw[dedge] (room2) -- (room2humModality) node[midway,sloped,above,rotate=180] {\edgetext{modality}};
  \draw[dedge] (room2h point) -- (room2humModality) node[midway,sloped,above] {\edgetext{provides}};
  
  \node[modality] (room2occModality) at ($(room2)!0.5!(room2o point)$) {\modalitytext{occupancy}};
  \draw[dedge] (room2) -- (room2occModality) node[midway,sloped,above] {\edgetext{modality}};
  \draw[dedge] (room2o point) -- (room2occModality) node[midway,sloped,above] {\edgetext{provides}};
}

\newcommand{\genFloor}[0]{
  \draw[dbedge] (room1) -- (room2) node[midway,sloped,above] {\edgetext{adjacent}};
  
  \node[room] (floor) at ([yshift=14mm] $(room1)!0.5!(room2)$) {\nodetext{Floor}};
  \draw[dedge] (floor) -- (room1) node[midway,sloped,above] {\edgetext{contains}};
  \draw[dedge] (floor) -- (room2) node[midway,sloped,above] {\edgetext{contains}};
  
  \node[field] (floorName) at ([yshift=\datadist+6mm] floor.center) {\fieldtext{3rd}};
  \draw[dedge] (floor) -- (floorName) node[midway,sloped,above] {\edgetext{name}};
}



          
          \genRoomI{-\centerspacing}{0}{highlight:point,highlight:data,highlight:unit}
          \genRoomII{\centerspacing}{0}{highlight:point,highlight:data,highlight:unit}
          \genFloor
        \end{tikzpicture}
      }
    }
    \caption{Match sites of unit query.}
    \label{fig:topics:info:match:unit}
  \end{subfigure}
  \hfill
  \begin{subfigure}[b]{0.48\textwidth}
    \rotatebox{90}{
      \scalebox{0.5}{
        \begin{tikzpicture}[remember picture]
          \newcommand{\nodetext}[1]{\scalebox{0.7}{#1}}
\newcommand{\hlnodetext}[1]{\nodetext{\textbf{#1}}}
\newcommand{\modalitytext}[1]{\scalebox{0.65}{#1}}
\newcommand{\hlmodalitytext}[1]{\modalitytext{\textbf{#1}}}
\newcommand{\fieldtext}[1]{\scalebox{0.7}{#1}}
\newcommand{\hlfieldtext}[1]{\fieldtext{\textbf{#1}}}
\newcommand{\edgetext}[1]{\scalebox{0.7}{\textit{#1}}}
\newcommand{\hledgetext}[1]{\edgetext{\textbf{#1}}}
\newcommand{\centerspacing}[0]{2.0}
\newcommand{\pointsize}[0]{(8mm)}
\newcommand{\datadist}[0]{(11mm)}
\newcommand{\pointdist}[0]{(\pointsize/2+9mm)}
\newcommand{\roomdist}[0]{(49mm)}
\newcommand{\sqrttwo}[0]{(1.41421356237)}
\newcommand{\diagside}[0]{(0.70710678119)}

\tikzstyle{dedge} = [->,>=stealth,draw=black]
\tikzstyle{dbedge} = [<->,>=stealth,draw=black]
\tikzstyle{node}=[
  overlay,
  circle,
  align=center,
  anchor=center,
]
\tikzstyle{point}=[
  node,
  draw=blue,
  minimum height=\pointsize,
  minimum width=\pointsize,
]
\tikzstyle{room}=[
  node,
  draw=blue,
  minimum height=\pointsize,
  minimum width=\pointsize,
]
\tikzstyle{modality}=[
  node,
  draw=purple,
]
\tikzstyle{unit}=[
  node,
  draw=teal,
  anchor=center,
]
\tikzstyle{data}=[
  node,
  draw=teal,
  anchor=center,
]
\tikzstyle{field}=[
  data,
  rectangle,
  anchor=center,
]
\tikzstyle{highlight}=[
  very thick,
]
\tikzstyle{highlighted-point}=[
  point,
  highlight,
  fill=blue!10,
]
\tikzstyle{highlighted-room}=[
  room,
  highlight,
  fill=blue!10,
]
\tikzstyle{highlighted-modality}=[
  modality,
  highlight,
  fill=purple!10,
]
\tikzstyle{highlighted-unit}=[
  unit,
  highlight,
  fill=teal!10,
]
\tikzstyle{highlighted-data}=[
  data,
  highlight,
  fill=teal!10,
]
\tikzstyle{highlighted-field}=[
  field,
  highlight,
  fill=teal!10,
]

\newcommand{\data}[6]{
  \IfSubStr{#6}{highlight:data}{
    \node[highlighted-data] (#2) at (#3) {\hlnodetext{data}};
    \node[highlighted-field] (hist) at ([yshift=-\datadist,xshift=-1*#1*\datadist] #2.center) {\hlfieldtext{#4}};
    \node[highlighted-field] (live) at ([yshift=-\datadist,xshift= 1*#1*\datadist] #2.center) {\hlfieldtext{#5}};
    \draw[dedge,highlight] (#2) -- (hist) node[midway,sloped,above] {\hledgetext{hist}};
    \draw[dedge,highlight] (#2) -- (live) node[midway,sloped,above] {\hledgetext{live}};
  }{
    \node[data] (#2) at (#3) {\nodetext{data}};
    \node[field] (hist) at ([yshift=-\datadist,xshift=-1*#1*\datadist] #2.center) {\fieldtext{#4}};
    \node[field] (live) at ([yshift=-\datadist,xshift= 1*#1*\datadist] #2.center) {\fieldtext{#5}};
    \draw[dedge] (#2) -- (hist) node[midway,sloped,above] {\edgetext{hist}};
    \draw[dedge] (#2) -- (live) node[midway,sloped,above] {\edgetext{live}};
  }
}
\renewcommand{\unit}[4]{
  \IfSubStr{#4}{highlight:unit}{
    \node[highlighted-unit] (#1) at (#2) {\hlnodetext{\si{#3}}};
  }{
    \node[unit] (#1) at (#2) {\nodetext{\si{#3}}};
  }
}
\newcommand{\point}[8][1]{
  \IfSubStr{#8}{highlight:point}{
    \node[highlighted-point] (#2 point) at (#3) {\hlnodetext{#4}};
  }{
    \node[point] (#2 point) at (#3) {\nodetext{#4}};
  }
  \data{#1}{data}{[yshift=-\pointdist,xshift=-1*#1*\pointdist] #2 point.center}{#6}{#7}{#8};
  \unit{unit}{    [yshift=-\pointdist,xshift=   #1*\pointdist] #2 point.center}{#5}{#8};
  \IfSubStr{#8}{highlight:unit}{
    \draw[dedge,highlight] (#2 point) -- (unit) node[midway,sloped,above] {\hledgetext{unit}};
  }{
    \draw[dedge] (#2 point) -- (unit) node[midway,sloped,above] {\edgetext{unit}};
  }
  \IfSubStr{#8}{highlight:data}{
    \draw[dedge,highlight] (#2 point) -- (data) node[midway,sloped,above] {\hledgetext{data}};
  }{
    \draw[dedge] (#2 point) -- (data) node[midway,sloped,above] {\edgetext{data}};
  }
}

\newcommand{\genRoomI}[3]{
  \node[room] (room1) at (#1, #2) {\nodetext{Room}};
  \node[field] (room1area) at ([yshift=\datadist+5mm] room1.center) {\fieldtext{17 \si{\square\meter}}};
  \draw[dedge] (room1) -- (room1area) node[midway,sloped,above] {\edgetext{area}};
  
  \point{room1t}{[xshift=-\roomdist*\diagside,yshift=-\roomdist*\diagside] room1}{TC}{\degreeCelsius}{12}{17c5ae15}{#3}
  \point{room1h}{[xshift=0,yshift=-\roomdist] room1}{Hum}{\percent}{13}{65d0be73}{}
  \point{room1o}{[xshift=-\roomdist,yshift=0] room1}{PIR}{}{24}{5a3573c3}{}
  
  \node[modality] (room1tempModality) at ($(room1)!0.5!(room1t point)$) {\modalitytext{temperature}};
  \draw[dedge] (room1) -- (room1tempModality) node[midway,sloped,above] {\edgetext{modality}};
  \draw[dedge] (room1t point) -- (room1tempModality) node[midway,sloped,above] {\edgetext{provides}};
  
  \node[modality] (room1humModality) at ($(room1)!0.5!(room1h point)$) {\modalitytext{relative}\\\modalitytext{humidity}};
  \draw[dedge] (room1) -- (room1humModality) node[midway,sloped,above,rotate=180] {\edgetext{modality}};
  \draw[dedge] (room1h point) -- (room1humModality) node[midway,sloped,above] {\edgetext{provides}};
  
  \node[modality] (room1occModality) at ($(room1)!0.5!(room1o point)$) {\modalitytext{occupancy}};
  \draw[dedge] (room1) -- (room1occModality) node[midway,sloped,above] {\edgetext{modality}};
  \draw[dedge] (room1o point) -- (room1occModality) node[midway,sloped,above] {\edgetext{provides}};
}

\newcommand{\genRoomII}[3]{
  \node[room] (room2) at (#1, #2) {\nodetext{Room}};
  \node[field] (room2area) at ([yshift=\datadist+5mm] room2.center) {\fieldtext{23 \si{\square\meter}}};
  \draw[dedge] (room2) -- (room2area) node[midway,sloped,above] {\edgetext{area}};
  
  \point[-1]{room2t}{[xshift=\roomdist*\diagside,yshift=-\roomdist*\diagside] room2}{TM}{\degreeCelsius}{16}{3b83ce7c}{}
  \point[-1]{room2h}{[xshift=0,yshift=-\roomdist] room2}{Hum}{\percent}{53}{91729592}{}
  \point[-1]{room2o}{[xshift=\roomdist,yshift=0] room2}{PIR}{}{27}{b4ae871d}{}
  
  \node[modality] (room2tempModality) at ($(room2)!0.5!(room2t point)$) {\modalitytext{temperature}};
  \draw[dedge] (room2) -- (room2tempModality) node[midway,sloped,above] {\edgetext{modality}};
  \draw[dedge] (room2t point) -- (room2tempModality) node[midway,sloped,above] {\edgetext{provides}};
  
  \node[modality] (room2humModality) at ($(room2)!0.5!(room2h point)$) {\modalitytext{relative}\\\modalitytext{humidity}};
  \draw[dedge] (room2) -- (room2humModality) node[midway,sloped,above,rotate=180] {\edgetext{modality}};
  \draw[dedge] (room2h point) -- (room2humModality) node[midway,sloped,above] {\edgetext{provides}};
  
  \node[modality] (room2occModality) at ($(room2)!0.5!(room2o point)$) {\modalitytext{occupancy}};
  \draw[dedge] (room2) -- (room2occModality) node[midway,sloped,above] {\edgetext{modality}};
  \draw[dedge] (room2o point) -- (room2occModality) node[midway,sloped,above] {\edgetext{provides}};
}

\newcommand{\genFloor}[0]{
  \draw[dbedge] (room1) -- (room2) node[midway,sloped,above] {\edgetext{adjacent}};
  
  \node[room] (floor) at ([yshift=14mm] $(room1)!0.5!(room2)$) {\nodetext{Floor}};
  \draw[dedge] (floor) -- (room1) node[midway,sloped,above] {\edgetext{contains}};
  \draw[dedge] (floor) -- (room2) node[midway,sloped,above] {\edgetext{contains}};
  
  \node[field] (floorName) at ([yshift=\datadist+6mm] floor.center) {\fieldtext{3rd}};
  \draw[dedge] (floor) -- (floorName) node[midway,sloped,above] {\edgetext{name}};
}



          
          \genRoomI{-\centerspacing}{0}{highlight:point,highlight:data,highlight:unit}
          \genRoomII{\centerspacing}{0}{highlight:point,highlight:data,highlight:unit}
          \genFloor
        \end{tikzpicture}
      }
    }
    \caption{Match sites of data query.}
    \label{fig:topics:info:match:data}
  \end{subfigure}
  
  \caption[Example of match sites in an abstract information model]{Example of match sites in an abstract information model.}
  \label{fig:topics:info:match}
\end{figure}

\subsection{Result Sets}

% text

% fig

\subsection{Result Sets}

\subsection{Model Reuse and Extension}

% note that the same model can be used for other applications, currently you could calculate the per-floor fractional occupancy, but the model could be extended

% list of examples of extensions: polygons for rooms and floors, physical location of sensors, actuators, electrical distribution trees, who rents/occupies the rooms (and their climate preferences)

\subsection{Building Abstractions}

% automatic stacking of virtual sensors: suite of virtual sensors, querying for relative humidity and matching model+stack of virtual sensors

% automatic unit conversion

\subsection{Ontologies}
\subsubsection{RDF}
\subsubsection{OWL}
\subsubsection{Select Ontologies}

% brick
\idx{Brick}\url{https://brickschema.org}

% web of things

% schema.org

% QUDT
The QUDT\idx{QUDT}\footnote{\url{https://www.qudt.org}} ontology models units. While it is the de-facto ontology for this, it is not particularly convenient for automatic conversion.

\subsection{Property Graphs}
\subsubsection{Neo4J}

\idx{Neo4j}\url{https://neo4j.com}

\subsection{Query Languages}
\subsubsection{Cypher}
\subsubsection{OpenCypher}

\idx{OpenCypher}\url{https://opencypher.org}

\subsubsection{Gremlin}
\subsubsection{SparQL}


