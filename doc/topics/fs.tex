\section{Filesystems}

There are many types of filesystems\idx{Filesystem}:
\begin{itemize}
  \descitem{Log Filesystem}
  \descitem{Object Store}
  \descitem{General-purpose filesystem}
\end{itemize}

% clarification
In this section we are going to cover general-purpose filesystems. This is what most people thing about when they hear the term filesystem, and it is perfectly fair to use the term filesystem as equivant to this.

Such a filesystem \textsl{organizes} files. The forms of supported organization varies from flat filesystems, over hierarchical filesystems to what is essentially graph filesystems. The organization type refers to the underlying data structure:
\begin{itemize}
  \descitem{Flat Filesystem}\idxx{Filesystem!Flat}{Flat filesystem} A file can be named through an ID without a context.
  \descitem{Hierarchical Filesystem}\idxx{Filesystem!Hierarchical}{Hierarchical filesystem} A file can be named through an ID given a context that represents a position in a tree structure.
  \descitem{Graph-Organized Filesystem}\idxx{Filesystem!Graph-organized}{Graph-organized filesystem} This is a hierarchical filesystem with links (that violate the tree property). Accordingly, you may encounter cycles in such filesystems.
\end{itemize}

\subsection{Files}

\subsection{Structure}

\subsection{Paths}

\subsection{Mountpoints}

\subsection{Devices}

\subsection{Inotify}

\subsection{FUSE}

\subsection{Qualities}

\subsubsection{Snapshots}

\subsubsection{Transactional}

\subsection{Interaction Examples}

\subsubsection{Copying a File}

\subsubsection{Editing a File}

