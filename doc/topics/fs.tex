\input{include/fs.tex}

\section{Filesystems}

There are many types of filesystems\idx{Filesystem}:
\begin{itemize}
  \descitem{Log Filesystem}
  \descitem{Object Store}
  \descitem{General-purpose filesystem}
\end{itemize}

% clarification
In this section we are going to cover general-purpose filesystems. This is what most people thing about when they hear the term filesystem, and it is perfectly fair to use the term filesystem as equivant to this.

Such a filesystem \textsl{organizes} files. The forms of supported organization varies from flat filesystems, over hierarchical filesystems to what is essentially graph filesystems. The organization type refers to the underlying data structure:
\begin{itemize}
  \descitem{Flat Filesystem}\idxx{Filesystem!Flat}{Flat filesystem} A file can be named through an ID without a context.
  \descitem{Hierarchical Filesystem}\idxx{Filesystem!Hierarchical}{Hierarchical filesystem} A file can be named through an ID given a context that represents a position in a tree structure.
  \descitem{Graph-Organized Filesystem}\idxx{Filesystem!Graph-organized}{Graph-organized filesystem} This is a hierarchical filesystem with links (that violate the tree property). Accordingly, you may encounter cycles in such filesystems.
\end{itemize}

\subsection{Files}

% definition: metadata, access rights, contents (byte sequence)
\idx{File}
The notion of a file change a bit from filesystem to filesystem. However, it usually covers:
\begin{itemize}
  \descitem{Contents} A sequence of bytes.
  \descitem{Metadata} \idxx{File!Metadata}{Metadata} Covering:
    \begin{itemize}
      \descitem{Ownership} \idxx{File!Ownership}{Ownership} Who \textsl{owns} the file (e.g., for quota\idx{Quota} purposes).
      \descitem{Rights} \idxx{File!Rights}{Rights} For assigning users, or groups of users, rights to perform certain operations on the file (e.g., read, write and execute).
      \descitem{Timestamps} \idxx{File!Timestamps}{Timestamps} E.g., for creation, modification and access.
    \end{itemize}
\end{itemize}

% operating on: pointer, read & write number of characters, seeking, end-of-file
While APIs may vary, the general model of interacting with a file is that the contents is an array of bytes, and when you open that file you get a point to the first element. You can then read or write (up to) some number of bytes from this position onwards. The pointer automatically advances, but it is also possible to perform arbitrary adjustments to this point by performing a \textsl{seek}\idx{Seek operation} operation. Reading beyond the largest index allocated in the file will result in reading the EOF\idx{End of file} (end-of-file) character.

% flushing: consequences of not buffering, layers of buffers, flushing

\subsection{Structure}

% directories: a special kind of file, a collection of files with unique names

\begin{figure}[tbp]
  \begin{center}
    \begin{tikzpicture}[remember picture]
      \genCoordsSDF{}
      \genFsI{}
    \end{tikzpicture}
  \end{center}
  \caption{Filesystem tree structure}
  \label{fig:topics:fs:dir}
\end{figure}

% mountpoints

\begin{figure}[tbp]
  \begin{center}
    \begin{tikzpicture}[remember picture]
      \genCoordsSDF{}
      \genFsI{}
      \genFsII{}
      \genMount{}
    \end{tikzpicture}
  \end{center}
  \caption{Mounting of one filesystem on another.}
  \label{fig:topics:fs:mount}
\end{figure}

% links: a named reference

\begin{figure}[tbp]
  \begin{center}
    \begin{tikzpicture}[remember picture]
      \genCoordsSDF{}
      \genFsI{}
      \genFsII{}
      \genMount{}
      \genLinkI{}
      \genLinkII{}
    \end{tikzpicture}
  \end{center}
  \caption[Filesystem links]{Filesystem links. Note that one link goes backward, thereby creating a cycle in the graph, while the other does not.}
  \label{fig:topics:fs:links}
\end{figure}

% why is it faster to move a file than it is to copy it?

\subsection{Paths}

% absolute path

\begin{figure}[tbp]
  \begin{center}
    \begin{tikzpicture}[remember picture]
      \genCoordsSDF{}
      \genFsI{highlight:rel,highlight:abs}
      \genFsII{highlight}
      \genMount{highlight}
      \genLinkI{}
      \genLinkII{}
    \end{tikzpicture}
  \end{center}
  \caption{Absolute path.}
  \label{fig:topics:fs:path:abs}
\end{figure}

% relative path: context of a working directory

\begin{figure}[tbp]
  \begin{center}
    \begin{tikzpicture}[remember picture]
      \genCoordsSDF{}
      \genFsI{highlight:rel}
      \genFsII{highlight}
      \genMount{highlight}
      \genLinkI{}
      \genLinkII{}
      \genWD{}
    \end{tikzpicture}
  \end{center}
  \caption{Relative path.}
  \label{fig:topics:fs:path:rel}
\end{figure}

% qualified path

\subsection{Devices}

% what do you do with a file: open, close, read, write, seek

% devices are the same, and are (on a UNIX system) represented as files

\subsection{Inotify}

% problem of polling

% event subscription

\subsection{FUSE}

\subsection{Qualities}

\subsubsection{Snapshots}

\subsubsection{Transactional}

\subsubsection{Copy on Write}

\subsection{Interaction Examples}

\subsubsection{Copying a File}

% detail example: how is a file copied? go through process

\begin{figure}[tbp]
  \begin{center}
    \begin{tikzpicture}
      \tikzstyle{title} = [
        anchor=south,
        align=center,
      ]
      \tikzstyle{arrow} = [
        thick,
        ->,
        >=stealth,
        draw=black,
      ]
      \tikzstyle{fileline}=[
        arrow,
        ultra thick,
        draw=black!20,
      ]
      \tikzstyle{fd} = [
        anchor=center,
        align=center,
        fill=black!20,
        minimum width=2cm,
        minimum height=0.6cm,
      ]
      \tikzstyle{data} = [
        anchor=center,
        fill=black,
        minimum width=1.5mm,
        minimum height=1.5mm,
      ]
      
      \newcommand{\shiftdistance}[0]{5cm}
      \newcommand{\figheight}[0]{5.2cm}
      
      \coordinate (memAnchor) at (0,-1cm);
      \coordinate (srcAnchor) at ([xshift=-\shiftdistance, yshift=0] memAnchor);
      \coordinate (dstAnchor) at ([xshift= \shiftdistance, yshift=0] memAnchor);
      
      % src line
      \node[title] (srcTitle) at ([yshift=2mm] srcAnchor) {Source\\File};
      \draw[fileline] (srcAnchor) -- ([yshift=-\figheight] srcAnchor);
      
      % dst line
      \node[title] (dstTitle) at ([yshift=2mm] dstAnchor) {Destination\\File};
      \draw[fileline] (dstAnchor) -- ([yshift=-\figheight] dstAnchor);
      
      % mem line
      \node[title] (memTitle) at ([yshift=2mm] memAnchor) {Program\\Memory};
      \draw[fileline] (memAnchor) -- ([yshift=-\figheight] memAnchor);
      
      % open(src)
      \node[fd] (srcOpen) at ([yshift=-4mm] srcAnchor) {\small{open file}};
      
      % open(dst)
      \node[fd] (dstOpen) at ([yshift=-11mm] dstAnchor) {\small{open fil}};
      
      % buffer
      \node[fd, minimum width=0.6cm, minimum height=1.4cm] (buffer) at ([yshift=-25mm] memAnchor) {\rotatebox{90}{\small{buffer}}};
      
      % read
      {
        \node[data] (srcDataI)   at ([yshift=-19mm] srcAnchor) {};
        \node[data] (srcDataII)  at ([yshift=-3.2mm] srcDataI) {};
        \node[data] (srcDataIII) at ([yshift=-3.2mm] srcDataII) {};
        
        \draw[arrow, dashed, draw=blue]
          ([xshift=4mm] srcDataII.center)--([xshift=\shiftdistance-0.3cm] srcDataII.center)
          node[midway,sloped,above]
          {\scalebox{0.7}{\textsl{\textcolor{blue}{read}}}};
      }
      
      % write
      {
        \node[data] (dstDataIII) at ([yshift=-31mm] dstAnchor) {};
        \node[data] (dstDataII)  at ([yshift=3.2mm] dstDataIII) {};
        \node[data] (dstDataI)   at ([yshift=3.2mm] dstDataII) {};
        
        \draw[arrow, dashed, draw=blue]
          ([xshift=-\shiftdistance+0.3cm] dstDataII.center)--([xshift=-4mm] dstDataII.center)
          node[midway,sloped,above]
          {\scalebox{0.7}{\textsl{\textcolor{blue}{write}}}};
      }
      
      % repetition
      \node[rectangle, thick, dashed, draw=purple, anchor=center, minimum width=2*\shiftdistance+10mm, minimum height=1.8cm] (repetition) at ([yshift=-25mm] memAnchor) {};
      \node[] (repetitionLabel) at ([xshift=-2mm] repetition.west) {\rotatebox{90}{\scalebox{0.60}{\textsl{\textcolor{purple}{repeat as needed}}}}};
      
      % close(src)
      \node[fd] (srcOpen) at ([yshift=-39mm] srcAnchor) {\small{close fil}};
      
      % close(dst)
      \node[fd] (dstOpen) at ([yshift=-46mm] dstAnchor) {\small{close fil}};
    \end{tikzpicture}
  \end{center}
  \caption{Flow followed when copying a file.}
  \label{fig:topics:fs:interaction:copy}
\end{figure}

\subsubsection{Editing a File}

Figure \ref{fig:topics:fs:interaction:edit} illustrates how a typical text editor with work with a file. First the file is opened. Then, the contents of the file is read and written to main memory. The user may then manipulate the in-memory representation through some user interface, and saved the modified buffer (memory representation) to disk. Finally, the program will close the file.

\begin{figure}[tbp]
  \begin{center}
    \begin{tikzpicture}
      \tikzstyle{title} = [
        anchor=south,
        align=center,
      ]
      \tikzstyle{arrow} = [
        thick,
        ->,
        >=stealth,
        draw=black,
      ]
      \tikzstyle{fileline}=[
        arrow,
        ultra thick,
        draw=black!20,
      ]
      \tikzstyle{fd} = [
        anchor=center,
        align=center,
        fill=black!20,
        minimum width=2cm,
        minimum height=0.6cm,
      ]
      \tikzstyle{data} = [
        anchor=center,
        fill=black,
        minimum width=1.5mm,
        minimum height=1.5mm,
      ]
      
      \newcommand{\shiftdistance}[0]{5cm}
      \newcommand{\moddistance}[0]{6mm}
      \newcommand{\figheight}[0]{5.2cm}
      
      \coordinate (memAnchor) at (0,-1cm);
      \coordinate (userAnchor) at ([xshift=-\shiftdistance, yshift=0] memAnchor);
      \coordinate (fileAnchor) at ([xshift= \shiftdistance, yshift=0] memAnchor);
      
      % user line
      \node[title] (userTitle) at ([yshift=2mm] userAnchor) {User\\Interface};
      \draw[fileline] (userAnchor) -- ([yshift=-\figheight] userAnchor);
      
      % file line
      \node[title] (fileTitle) at ([yshift=2mm] fileAnchor) {File of\\Interest};
      \draw[fileline] (fileAnchor) -- ([yshift=-\figheight] fileAnchor);
      
      % mem line
      \node[title] (memTitle) at ([yshift=2mm] memAnchor) {Program\\Memory};
      \draw[fileline] (memAnchor) -- ([yshift=-\figheight] memAnchor);
      
      % open(file)
      \node[fd] (fileOpen) at ([yshift=-4mm] fileAnchor) {\small{open file}};
      
      % buffer
      \node[fd, minimum width=0.6cm, minimum height=2.8cm] (buffer) at ([yshift=-25mm] memAnchor) {\rotatebox{90}{\small{buffer}}};
      
      % read
      {
        \node[data, anchor=north] (dataI)   at ([yshift=-2mm] fileOpen.south) {};
        \node[data] (dataII)  at ([yshift=-3.2mm] dataI) {};
        \node[data] (dataIII) at ([yshift=-3.2mm] dataII) {};
        
        \draw[arrow, dashed, draw=blue]
          ([xshift=-4mm] dataII.center)--([xshift=-\shiftdistance+0.3cm] dataII.center)
          node[midway,sloped,above]
          {\scalebox{0.7}{\textsl{\textcolor{blue}{read}}}};
      }
      
      % modify 1
      {
        \draw[arrow, dashed, draw=blue]
          ([xshift=-\shiftdistance, yshift=\moddistance] buffer.center)--([xshift=-0.3cm, yshift=\moddistance] buffer.center)
          node[midway,sloped,above]
          {\scalebox{0.7}{\textsl{\textcolor{blue}{modify}}}};
      }
      
      % modify 2
      {
        \draw[arrow, dashed, draw=blue]
          ([xshift=-\shiftdistance, yshift=0mm] buffer.center)--([xshift=-0.3cm, yshift=0mm] buffer.center)
          node[midway,sloped,above]
          {\scalebox{0.7}{\textsl{\textcolor{blue}{modify}}}};
      }
      
      % modify 3
      {
        \draw[arrow, dashed, draw=blue]
          ([xshift=-\shiftdistance, yshift=-\moddistance] buffer.center)--([xshift=-0.3cm, yshift=-\moddistance] buffer.center)
          node[midway,sloped,above]
          {\scalebox{0.7}{\textsl{\textcolor{blue}{modify}}}};
      }
      
      % write
      {
        \node[data] (dstDataIII) at ([yshift=-40mm] fileAnchor) {};
        \node[data] (dstDataII)  at ([yshift=3.2mm] dstDataIII) {};
        \node[data] (dstDataI)   at ([yshift=3.2mm] dstDataII) {};
        
        \draw[arrow, dashed, draw=blue]
          ([xshift=-\shiftdistance+0.3cm] dstDataII.center)--([xshift=-4mm] dstDataII.center)
          node[midway,sloped,above]
          {\scalebox{0.7}{\textsl{\textcolor{blue}{write}}}};
      }
      
      % close(file)
      \node[fd] (fileOpen) at ([yshift=-46mm] fileAnchor) {\small{close file}};
    \end{tikzpicture}
  \end{center}
  \caption{Flow followed by a text editor application.}
  \label{fig:topics:fs:interaction:edit}
\end{figure}



\section{Filesystems}

There are many types of filesystems\idx{Filesystem}:
\begin{itemize}
  \descitem{Log Filesystem}
  \descitem{Object Store}
  \descitem{General-purpose filesystem}
\end{itemize}

% clarification
In this section we are going to cover general-purpose filesystems. This is what most people thing about when they hear the term filesystem, and it is perfectly fair to use the term filesystem as equivant to this.

Such a filesystem \textsl{organizes} files. The forms of supported organization varies from flat filesystems, over hierarchical filesystems to what is essentially graph filesystems. The organization type refers to the underlying data structure:
\begin{itemize}
  \descitem{Flat Filesystem}\idxx{Filesystem!Flat}{Flat filesystem} A file can be named through an ID without a context.
  \descitem{Hierarchical Filesystem}\idxx{Filesystem!Hierarchical}{Hierarchical filesystem} A file can be named through an ID given a context that represents a position in a tree structure.
  \descitem{Graph-Organized Filesystem}\idxx{Filesystem!Graph-organized}{Graph-organized filesystem} This is a hierarchical filesystem with links (that violate the tree property). Accordingly, you may encounter cycles in such filesystems.
\end{itemize}

\subsection{Files}

% definition: metadata, access rights, contents (byte sequence)
\idx{File}
The notion of a file change a bit from filesystem to filesystem. However, it usually covers:
\begin{itemize}
  \descitem{Contents} A sequence of bytes.
  \descitem{Metadata} \idxx{File!Metadata}{Metadata} Covering:
    \begin{itemize}
      \descitem{Ownership} \idxx{File!Ownership}{Ownership} Who \textsl{owns} the file (e.g., for quota\idx{Quota} purposes).
      \descitem{Rights} \idxx{File!Rights}{Rights} For assigning users, or groups of users, rights to perform certain operations on the file (e.g., read, write and execute).
      \descitem{Timestamps} \idxx{File!Timestamps}{Timestamps} E.g., for creation, modification and access.
    \end{itemize}
\end{itemize}

% operating on: pointer, read & write number of characters, seeking, end-of-file
While APIs may vary, the general model of interacting with a file is that the contents is an array of bytes, and when you open that file you get a point to the first element. You can then read or write (up to) some number of bytes from this position onwards. The pointer automatically advances, but it is also possible to perform arbitrary adjustments to this point by performing a \textsl{seek}\idx{Seek operation} operation. Reading beyond the largest index allocated in the file will result in reading the EOF\idx{End of file} (end-of-file) character.

% flushing: consequences of not buffering, layers of buffers, flushing

\subsection{Structure}

% directories: a special kind of file, a collection of files with unique names

\begin{figure}[tbp]
  \begin{center}
    \begin{tikzpicture}[remember picture]
      \genCoordsSDF{}
      \genFsI{}
    \end{tikzpicture}
  \end{center}
  \caption{Filesystem tree structure}
  \label{fig:topics:fs:dir}
\end{figure}

% mountpoints

\begin{figure}[tbp]
  \begin{center}
    \begin{tikzpicture}[remember picture]
      \genCoordsSDF{}
      \genFsI{}
      \genFsII{}
      \genMount{}
    \end{tikzpicture}
  \end{center}
  \caption{Mounting of one filesystem on another.}
  \label{fig:topics:fs:mount}
\end{figure}

% links: a named reference

\begin{figure}[tbp]
  \begin{center}
    \begin{tikzpicture}[remember picture]
      \genCoordsSDF{}
      \genFsI{}
      \genFsII{}
      \genMount{}
      \genLinkI{}
      \genLinkII{}
    \end{tikzpicture}
  \end{center}
  \caption[Filesystem links]{Filesystem links. Note that one link goes backward, thereby creating a cycle in the graph, while the other does not.}
  \label{fig:topics:fs:links}
\end{figure}

% why is it faster to move a file than it is to copy it?

\subsection{Paths}

% absolute path

\begin{figure}[tbp]
  \begin{center}
    \begin{tikzpicture}[remember picture]
      \genCoordsSDF{}
      \genFsI{highlight:rel,highlight:abs}
      \genFsII{highlight}
      \genMount{highlight}
      \genLinkI{}
      \genLinkII{}
    \end{tikzpicture}
  \end{center}
  \caption{Absolute path.}
  \label{fig:topics:fs:path:abs}
\end{figure}

% relative path: context of a working directory

\begin{figure}[tbp]
  \begin{center}
    \begin{tikzpicture}[remember picture]
      \genCoordsSDF{}
      \genFsI{highlight:rel}
      \genFsII{highlight}
      \genMount{highlight}
      \genLinkI{}
      \genLinkII{}
      \genWD{}
    \end{tikzpicture}
  \end{center}
  \caption{Relative path.}
  \label{fig:topics:fs:path:rel}
\end{figure}

% qualified path

\subsection{Devices}

% what do you do with a file: open, close, read, write, seek

% devices are the same, and are (on a UNIX system) represented as files

\subsection{Inotify}

% problem of polling

% event subscription

\subsection{FUSE}

\subsection{Qualities}

\subsubsection{Snapshots}

\subsubsection{Transactional}

\subsubsection{Copy on Write}

\subsection{Interaction Examples}

\subsubsection{Copying a File}

% detail example: how is a file copied? go through process

\begin{figure}[tbp]
  \begin{center}
    \begin{tikzpicture}
      \tikzstyle{title} = [
        anchor=south,
        align=center,
      ]
      \tikzstyle{arrow} = [
        thick,
        ->,
        >=stealth,
        draw=black,
      ]
      \tikzstyle{fileline}=[
        arrow,
        ultra thick,
        draw=black!20,
      ]
      \tikzstyle{fd} = [
        anchor=center,
        align=center,
        fill=black!20,
        minimum width=2cm,
        minimum height=0.6cm,
      ]
      \tikzstyle{data} = [
        anchor=center,
        fill=black,
        minimum width=1.5mm,
        minimum height=1.5mm,
      ]
      
      \newcommand{\shiftdistance}[0]{5cm}
      \newcommand{\figheight}[0]{5.2cm}
      
      \coordinate (memAnchor) at (0,-1cm);
      \coordinate (srcAnchor) at ([xshift=-\shiftdistance, yshift=0] memAnchor);
      \coordinate (dstAnchor) at ([xshift= \shiftdistance, yshift=0] memAnchor);
      
      % src line
      \node[title] (srcTitle) at ([yshift=2mm] srcAnchor) {Source\\File};
      \draw[fileline] (srcAnchor) -- ([yshift=-\figheight] srcAnchor);
      
      % dst line
      \node[title] (dstTitle) at ([yshift=2mm] dstAnchor) {Destination\\File};
      \draw[fileline] (dstAnchor) -- ([yshift=-\figheight] dstAnchor);
      
      % mem line
      \node[title] (memTitle) at ([yshift=2mm] memAnchor) {Program\\Memory};
      \draw[fileline] (memAnchor) -- ([yshift=-\figheight] memAnchor);
      
      % open(src)
      \node[fd] (srcOpen) at ([yshift=-4mm] srcAnchor) {\small{open file}};
      
      % open(dst)
      \node[fd] (dstOpen) at ([yshift=-11mm] dstAnchor) {\small{open fil}};
      
      % buffer
      \node[fd, minimum width=0.6cm, minimum height=1.4cm] (buffer) at ([yshift=-25mm] memAnchor) {\rotatebox{90}{\small{buffer}}};
      
      % read
      {
        \node[data] (srcDataI)   at ([yshift=-19mm] srcAnchor) {};
        \node[data] (srcDataII)  at ([yshift=-3.2mm] srcDataI) {};
        \node[data] (srcDataIII) at ([yshift=-3.2mm] srcDataII) {};
        
        \draw[arrow, dashed, draw=blue]
          ([xshift=4mm] srcDataII.center)--([xshift=\shiftdistance-0.3cm] srcDataII.center)
          node[midway,sloped,above]
          {\scalebox{0.7}{\textsl{\textcolor{blue}{read}}}};
      }
      
      % write
      {
        \node[data] (dstDataIII) at ([yshift=-31mm] dstAnchor) {};
        \node[data] (dstDataII)  at ([yshift=3.2mm] dstDataIII) {};
        \node[data] (dstDataI)   at ([yshift=3.2mm] dstDataII) {};
        
        \draw[arrow, dashed, draw=blue]
          ([xshift=-\shiftdistance+0.3cm] dstDataII.center)--([xshift=-4mm] dstDataII.center)
          node[midway,sloped,above]
          {\scalebox{0.7}{\textsl{\textcolor{blue}{write}}}};
      }
      
      % repetition
      \node[rectangle, thick, dashed, draw=purple, anchor=center, minimum width=2*\shiftdistance+10mm, minimum height=1.8cm] (repetition) at ([yshift=-25mm] memAnchor) {};
      \node[] (repetitionLabel) at ([xshift=-2mm] repetition.west) {\rotatebox{90}{\scalebox{0.60}{\textsl{\textcolor{purple}{repeat as needed}}}}};
      
      % close(src)
      \node[fd] (srcOpen) at ([yshift=-39mm] srcAnchor) {\small{close fil}};
      
      % close(dst)
      \node[fd] (dstOpen) at ([yshift=-46mm] dstAnchor) {\small{close fil}};
    \end{tikzpicture}
  \end{center}
  \caption{Flow followed when copying a file.}
  \label{fig:topics:fs:interaction:copy}
\end{figure}

\subsubsection{Editing a File}

Figure \ref{fig:topics:fs:interaction:edit} illustrates how a typical text editor with work with a file. First the file is opened. Then, the contents of the file is read and written to main memory. The user may then manipulate the in-memory representation through some user interface, and saved the modified buffer (memory representation) to disk. Finally, the program will close the file.

\begin{figure}[tbp]
  \begin{center}
    \begin{tikzpicture}
      \tikzstyle{title} = [
        anchor=south,
        align=center,
      ]
      \tikzstyle{arrow} = [
        thick,
        ->,
        >=stealth,
        draw=black,
      ]
      \tikzstyle{fileline}=[
        arrow,
        ultra thick,
        draw=black!20,
      ]
      \tikzstyle{fd} = [
        anchor=center,
        align=center,
        fill=black!20,
        minimum width=2cm,
        minimum height=0.6cm,
      ]
      \tikzstyle{data} = [
        anchor=center,
        fill=black,
        minimum width=1.5mm,
        minimum height=1.5mm,
      ]
      
      \newcommand{\shiftdistance}[0]{5cm}
      \newcommand{\moddistance}[0]{6mm}
      \newcommand{\figheight}[0]{5.2cm}
      
      \coordinate (memAnchor) at (0,-1cm);
      \coordinate (userAnchor) at ([xshift=-\shiftdistance, yshift=0] memAnchor);
      \coordinate (fileAnchor) at ([xshift= \shiftdistance, yshift=0] memAnchor);
      
      % user line
      \node[title] (userTitle) at ([yshift=2mm] userAnchor) {User\\Interface};
      \draw[fileline] (userAnchor) -- ([yshift=-\figheight] userAnchor);
      
      % file line
      \node[title] (fileTitle) at ([yshift=2mm] fileAnchor) {File of\\Interest};
      \draw[fileline] (fileAnchor) -- ([yshift=-\figheight] fileAnchor);
      
      % mem line
      \node[title] (memTitle) at ([yshift=2mm] memAnchor) {Program\\Memory};
      \draw[fileline] (memAnchor) -- ([yshift=-\figheight] memAnchor);
      
      % open(file)
      \node[fd] (fileOpen) at ([yshift=-4mm] fileAnchor) {\small{open file}};
      
      % buffer
      \node[fd, minimum width=0.6cm, minimum height=2.8cm] (buffer) at ([yshift=-25mm] memAnchor) {\rotatebox{90}{\small{buffer}}};
      
      % read
      {
        \node[data, anchor=north] (dataI)   at ([yshift=-2mm] fileOpen.south) {};
        \node[data] (dataII)  at ([yshift=-3.2mm] dataI) {};
        \node[data] (dataIII) at ([yshift=-3.2mm] dataII) {};
        
        \draw[arrow, dashed, draw=blue]
          ([xshift=-4mm] dataII.center)--([xshift=-\shiftdistance+0.3cm] dataII.center)
          node[midway,sloped,above]
          {\scalebox{0.7}{\textsl{\textcolor{blue}{read}}}};
      }
      
      % modify 1
      {
        \draw[arrow, dashed, draw=blue]
          ([xshift=-\shiftdistance, yshift=\moddistance] buffer.center)--([xshift=-0.3cm, yshift=\moddistance] buffer.center)
          node[midway,sloped,above]
          {\scalebox{0.7}{\textsl{\textcolor{blue}{modify}}}};
      }
      
      % modify 2
      {
        \draw[arrow, dashed, draw=blue]
          ([xshift=-\shiftdistance, yshift=0mm] buffer.center)--([xshift=-0.3cm, yshift=0mm] buffer.center)
          node[midway,sloped,above]
          {\scalebox{0.7}{\textsl{\textcolor{blue}{modify}}}};
      }
      
      % modify 3
      {
        \draw[arrow, dashed, draw=blue]
          ([xshift=-\shiftdistance, yshift=-\moddistance] buffer.center)--([xshift=-0.3cm, yshift=-\moddistance] buffer.center)
          node[midway,sloped,above]
          {\scalebox{0.7}{\textsl{\textcolor{blue}{modify}}}};
      }
      
      % write
      {
        \node[data] (dstDataIII) at ([yshift=-40mm] fileAnchor) {};
        \node[data] (dstDataII)  at ([yshift=3.2mm] dstDataIII) {};
        \node[data] (dstDataI)   at ([yshift=3.2mm] dstDataII) {};
        
        \draw[arrow, dashed, draw=blue]
          ([xshift=-\shiftdistance+0.3cm] dstDataII.center)--([xshift=-4mm] dstDataII.center)
          node[midway,sloped,above]
          {\scalebox{0.7}{\textsl{\textcolor{blue}{write}}}};
      }
      
      % close(file)
      \node[fd] (fileOpen) at ([yshift=-46mm] fileAnchor) {\small{close file}};
    \end{tikzpicture}
  \end{center}
  \caption{Flow followed by a text editor application.}
  \label{fig:topics:fs:interaction:edit}
\end{figure}



\section{Filesystems}

There are many types of filesystems\idx{Filesystem}:
\begin{itemize}
  \descitem{Log Filesystem}
  \descitem{Object Store}
  \descitem{General-purpose filesystem}
\end{itemize}

% clarification
In this section we are going to cover general-purpose filesystems. This is what most people thing about when they hear the term filesystem, and it is perfectly fair to use the term filesystem as equivant to this.

Such a filesystem \textsl{organizes} files. The forms of supported organization varies from flat filesystems, over hierarchical filesystems to what is essentially graph filesystems. The organization type refers to the underlying data structure:
\begin{itemize}
  \descitem{Flat Filesystem}\idxx{Filesystem!Flat}{Flat filesystem} A file can be named through an ID without a context.
  \descitem{Hierarchical Filesystem}\idxx{Filesystem!Hierarchical}{Hierarchical filesystem} A file can be named through an ID given a context that represents a position in a tree structure.
  \descitem{Graph-Organized Filesystem}\idxx{Filesystem!Graph-organized}{Graph-organized filesystem} This is a hierarchical filesystem with links (that violate the tree property). Accordingly, you may encounter cycles in such filesystems.
\end{itemize}

\subsection{Files}

% definition: metadata, access rights, contents (byte sequence)
\idx{File}
The notion of a file change a bit from filesystem to filesystem. However, it usually covers:
\begin{itemize}
  \descitem{Contents} A sequence of bytes.
  \descitem{Metadata} \idxx{File!Metadata}{Metadata} Covering:
    \begin{itemize}
      \descitem{Ownership} \idxx{File!Ownership}{Ownership} Who \textsl{owns} the file (e.g., for quota\idx{Quota} purposes).
      \descitem{Rights} \idxx{File!Rights}{Rights} For assigning users, or groups of users, rights to perform certain operations on the file (e.g., read, write and execute).
      \descitem{Timestamps} \idxx{File!Timestamps}{Timestamps} E.g., for creation, modification and access.
    \end{itemize}
\end{itemize}

% operating on: pointer, read & write number of characters, seeking, end-of-file
While APIs may vary, the general model of interacting with a file is that the contents is an array of bytes, and when you open that file you get a point to the first element. You can then read or write (up to) some number of bytes from this position onwards. The pointer automatically advances, but it is also possible to perform arbitrary adjustments to this point by performing a \textsl{seek}\idx{Seek operation} operation. Reading beyond the largest index allocated in the file will result in reading the EOF\idx{End of file} (end-of-file) character.

% flushing: consequences of not buffering, layers of buffers, flushing

\subsection{Structure}

% directories: a special kind of file, a collection of files with unique names

\begin{figure}[tbp]
  \begin{center}
    \begin{tikzpicture}[remember picture]
      \genCoordsSDF{}
      \genFsI{}
    \end{tikzpicture}
  \end{center}
  \caption{Filesystem tree structure}
  \label{fig:topics:fs:dir}
\end{figure}

% mountpoints

\begin{figure}[tbp]
  \begin{center}
    \begin{tikzpicture}[remember picture]
      \genCoordsSDF{}
      \genFsI{}
      \genFsII{}
      \genMount{}
    \end{tikzpicture}
  \end{center}
  \caption{Mounting of one filesystem on another.}
  \label{fig:topics:fs:mount}
\end{figure}

% links: a named reference

\begin{figure}[tbp]
  \begin{center}
    \begin{tikzpicture}[remember picture]
      \genCoordsSDF{}
      \genFsI{}
      \genFsII{}
      \genMount{}
      \genLinkI{}
      \genLinkII{}
    \end{tikzpicture}
  \end{center}
  \caption[Filesystem links]{Filesystem links. Note that one link goes backward, thereby creating a cycle in the graph, while the other does not.}
  \label{fig:topics:fs:links}
\end{figure}

% why is it faster to move a file than it is to copy it?

\subsection{Paths}

% absolute path

\begin{figure}[tbp]
  \begin{center}
    \begin{tikzpicture}[remember picture]
      \genCoordsSDF{}
      \genFsI{highlight:rel,highlight:abs}
      \genFsII{highlight}
      \genMount{highlight}
      \genLinkI{}
      \genLinkII{}
    \end{tikzpicture}
  \end{center}
  \caption{Absolute path.}
  \label{fig:topics:fs:path:abs}
\end{figure}

% relative path: context of a working directory

\begin{figure}[tbp]
  \begin{center}
    \begin{tikzpicture}[remember picture]
      \genCoordsSDF{}
      \genFsI{highlight:rel}
      \genFsII{highlight}
      \genMount{highlight}
      \genLinkI{}
      \genLinkII{}
      \genWD{}
    \end{tikzpicture}
  \end{center}
  \caption{Relative path.}
  \label{fig:topics:fs:path:rel}
\end{figure}

% qualified path

\subsection{Devices}

% what do you do with a file: open, close, read, write, seek

% devices are the same, and are (on a UNIX system) represented as files

\subsection{Inotify}

% problem of polling

% event subscription

\subsection{FUSE}

\subsection{Qualities}

\subsubsection{Snapshots}

\subsubsection{Transactional}

\subsubsection{Copy on Write}

\subsection{Interaction Examples}

\subsubsection{Copying a File}

% detail example: how is a file copied? go through process

\begin{figure}[tbp]
  \begin{center}
    \begin{tikzpicture}
      \tikzstyle{title} = [
        anchor=south,
        align=center,
      ]
      \tikzstyle{arrow} = [
        thick,
        ->,
        >=stealth,
        draw=black,
      ]
      \tikzstyle{fileline}=[
        arrow,
        ultra thick,
        draw=black!20,
      ]
      \tikzstyle{fd} = [
        anchor=center,
        align=center,
        fill=black!20,
        minimum width=2cm,
        minimum height=0.6cm,
      ]
      \tikzstyle{data} = [
        anchor=center,
        fill=black,
        minimum width=1.5mm,
        minimum height=1.5mm,
      ]
      
      \newcommand{\shiftdistance}[0]{5cm}
      \newcommand{\figheight}[0]{5.2cm}
      
      \coordinate (memAnchor) at (0,-1cm);
      \coordinate (srcAnchor) at ([xshift=-\shiftdistance, yshift=0] memAnchor);
      \coordinate (dstAnchor) at ([xshift= \shiftdistance, yshift=0] memAnchor);
      
      % src line
      \node[title] (srcTitle) at ([yshift=2mm] srcAnchor) {Source\\File};
      \draw[fileline] (srcAnchor) -- ([yshift=-\figheight] srcAnchor);
      
      % dst line
      \node[title] (dstTitle) at ([yshift=2mm] dstAnchor) {Destination\\File};
      \draw[fileline] (dstAnchor) -- ([yshift=-\figheight] dstAnchor);
      
      % mem line
      \node[title] (memTitle) at ([yshift=2mm] memAnchor) {Program\\Memory};
      \draw[fileline] (memAnchor) -- ([yshift=-\figheight] memAnchor);
      
      % open(src)
      \node[fd] (srcOpen) at ([yshift=-4mm] srcAnchor) {\small{open file}};
      
      % open(dst)
      \node[fd] (dstOpen) at ([yshift=-11mm] dstAnchor) {\small{open fil}};
      
      % buffer
      \node[fd, minimum width=0.6cm, minimum height=1.4cm] (buffer) at ([yshift=-25mm] memAnchor) {\rotatebox{90}{\small{buffer}}};
      
      % read
      {
        \node[data] (srcDataI)   at ([yshift=-19mm] srcAnchor) {};
        \node[data] (srcDataII)  at ([yshift=-3.2mm] srcDataI) {};
        \node[data] (srcDataIII) at ([yshift=-3.2mm] srcDataII) {};
        
        \draw[arrow, dashed, draw=blue]
          ([xshift=4mm] srcDataII.center)--([xshift=\shiftdistance-0.3cm] srcDataII.center)
          node[midway,sloped,above]
          {\scalebox{0.7}{\textsl{\textcolor{blue}{read}}}};
      }
      
      % write
      {
        \node[data] (dstDataIII) at ([yshift=-31mm] dstAnchor) {};
        \node[data] (dstDataII)  at ([yshift=3.2mm] dstDataIII) {};
        \node[data] (dstDataI)   at ([yshift=3.2mm] dstDataII) {};
        
        \draw[arrow, dashed, draw=blue]
          ([xshift=-\shiftdistance+0.3cm] dstDataII.center)--([xshift=-4mm] dstDataII.center)
          node[midway,sloped,above]
          {\scalebox{0.7}{\textsl{\textcolor{blue}{write}}}};
      }
      
      % repetition
      \node[rectangle, thick, dashed, draw=purple, anchor=center, minimum width=2*\shiftdistance+10mm, minimum height=1.8cm] (repetition) at ([yshift=-25mm] memAnchor) {};
      \node[] (repetitionLabel) at ([xshift=-2mm] repetition.west) {\rotatebox{90}{\scalebox{0.60}{\textsl{\textcolor{purple}{repeat as needed}}}}};
      
      % close(src)
      \node[fd] (srcOpen) at ([yshift=-39mm] srcAnchor) {\small{close fil}};
      
      % close(dst)
      \node[fd] (dstOpen) at ([yshift=-46mm] dstAnchor) {\small{close fil}};
    \end{tikzpicture}
  \end{center}
  \caption{Flow followed when copying a file.}
  \label{fig:topics:fs:interaction:copy}
\end{figure}

\subsubsection{Editing a File}

Figure \ref{fig:topics:fs:interaction:edit} illustrates how a typical text editor with work with a file. First the file is opened. Then, the contents of the file is read and written to main memory. The user may then manipulate the in-memory representation through some user interface, and saved the modified buffer (memory representation) to disk. Finally, the program will close the file.

\begin{figure}[tbp]
  \begin{center}
    \begin{tikzpicture}
      \tikzstyle{title} = [
        anchor=south,
        align=center,
      ]
      \tikzstyle{arrow} = [
        thick,
        ->,
        >=stealth,
        draw=black,
      ]
      \tikzstyle{fileline}=[
        arrow,
        ultra thick,
        draw=black!20,
      ]
      \tikzstyle{fd} = [
        anchor=center,
        align=center,
        fill=black!20,
        minimum width=2cm,
        minimum height=0.6cm,
      ]
      \tikzstyle{data} = [
        anchor=center,
        fill=black,
        minimum width=1.5mm,
        minimum height=1.5mm,
      ]
      
      \newcommand{\shiftdistance}[0]{5cm}
      \newcommand{\moddistance}[0]{6mm}
      \newcommand{\figheight}[0]{5.2cm}
      
      \coordinate (memAnchor) at (0,-1cm);
      \coordinate (userAnchor) at ([xshift=-\shiftdistance, yshift=0] memAnchor);
      \coordinate (fileAnchor) at ([xshift= \shiftdistance, yshift=0] memAnchor);
      
      % user line
      \node[title] (userTitle) at ([yshift=2mm] userAnchor) {User\\Interface};
      \draw[fileline] (userAnchor) -- ([yshift=-\figheight] userAnchor);
      
      % file line
      \node[title] (fileTitle) at ([yshift=2mm] fileAnchor) {File of\\Interest};
      \draw[fileline] (fileAnchor) -- ([yshift=-\figheight] fileAnchor);
      
      % mem line
      \node[title] (memTitle) at ([yshift=2mm] memAnchor) {Program\\Memory};
      \draw[fileline] (memAnchor) -- ([yshift=-\figheight] memAnchor);
      
      % open(file)
      \node[fd] (fileOpen) at ([yshift=-4mm] fileAnchor) {\small{open file}};
      
      % buffer
      \node[fd, minimum width=0.6cm, minimum height=2.8cm] (buffer) at ([yshift=-25mm] memAnchor) {\rotatebox{90}{\small{buffer}}};
      
      % read
      {
        \node[data, anchor=north] (dataI)   at ([yshift=-2mm] fileOpen.south) {};
        \node[data] (dataII)  at ([yshift=-3.2mm] dataI) {};
        \node[data] (dataIII) at ([yshift=-3.2mm] dataII) {};
        
        \draw[arrow, dashed, draw=blue]
          ([xshift=-4mm] dataII.center)--([xshift=-\shiftdistance+0.3cm] dataII.center)
          node[midway,sloped,above]
          {\scalebox{0.7}{\textsl{\textcolor{blue}{read}}}};
      }
      
      % modify 1
      {
        \draw[arrow, dashed, draw=blue]
          ([xshift=-\shiftdistance, yshift=\moddistance] buffer.center)--([xshift=-0.3cm, yshift=\moddistance] buffer.center)
          node[midway,sloped,above]
          {\scalebox{0.7}{\textsl{\textcolor{blue}{modify}}}};
      }
      
      % modify 2
      {
        \draw[arrow, dashed, draw=blue]
          ([xshift=-\shiftdistance, yshift=0mm] buffer.center)--([xshift=-0.3cm, yshift=0mm] buffer.center)
          node[midway,sloped,above]
          {\scalebox{0.7}{\textsl{\textcolor{blue}{modify}}}};
      }
      
      % modify 3
      {
        \draw[arrow, dashed, draw=blue]
          ([xshift=-\shiftdistance, yshift=-\moddistance] buffer.center)--([xshift=-0.3cm, yshift=-\moddistance] buffer.center)
          node[midway,sloped,above]
          {\scalebox{0.7}{\textsl{\textcolor{blue}{modify}}}};
      }
      
      % write
      {
        \node[data] (dstDataIII) at ([yshift=-40mm] fileAnchor) {};
        \node[data] (dstDataII)  at ([yshift=3.2mm] dstDataIII) {};
        \node[data] (dstDataI)   at ([yshift=3.2mm] dstDataII) {};
        
        \draw[arrow, dashed, draw=blue]
          ([xshift=-\shiftdistance+0.3cm] dstDataII.center)--([xshift=-4mm] dstDataII.center)
          node[midway,sloped,above]
          {\scalebox{0.7}{\textsl{\textcolor{blue}{write}}}};
      }
      
      % close(file)
      \node[fd] (fileOpen) at ([yshift=-46mm] fileAnchor) {\small{close file}};
    \end{tikzpicture}
  \end{center}
  \caption{Flow followed by a text editor application.}
  \label{fig:topics:fs:interaction:edit}
\end{figure}



\section{Filesystems}

There are many types of filesystems\idx{Filesystem}:
\begin{itemize}
  \descitem{Log Filesystem}
  \descitem{Object Store}
  \descitem{General-purpose filesystem}
\end{itemize}

% clarification
In this section we are going to cover general-purpose filesystems. This is what most people thing about when they hear the term filesystem, and it is perfectly fair to use the term filesystem as equivant to this.

Such a filesystem \textsl{organizes} files. The forms of supported organization varies from flat filesystems, over hierarchical filesystems to what is essentially graph filesystems. The organization type refers to the underlying data structure:
\begin{itemize}
  \descitem{Flat Filesystem}\idxx{Filesystem!Flat}{Flat filesystem} A file can be named through an ID without a context.
  \descitem{Hierarchical Filesystem}\idxx{Filesystem!Hierarchical}{Hierarchical filesystem} A file can be named through an ID given a context that represents a position in a tree structure.
  \descitem{Graph-Organized Filesystem}\idxx{Filesystem!Graph-organized}{Graph-organized filesystem} This is a hierarchical filesystem with links (that violate the tree property). Accordingly, you may encounter cycles in such filesystems.
\end{itemize}

\subsection{Files}

% definition: metadata, access rights, contents (byte sequence)
\idx{File}
The notion of a file change a bit from filesystem to filesystem. However, it usually covers:
\begin{itemize}
  \descitem{Contents} A sequence of bytes.
  \descitem{Metadata} \idxx{File!Metadata}{Metadata} Covering:
    \begin{itemize}
      \descitem{Ownership} \idxx{File!Ownership}{Ownership} Who \textsl{owns} the file (e.g., for quota\idx{Quota} purposes).
      \descitem{Rights} \idxx{File!Rights}{Rights} For assigning users, or groups of users, rights to perform certain operations on the file (e.g., read, write and execute).
      \descitem{Timestamps} \idxx{File!Timestamps}{Timestamps} E.g., for creation, modification and access.
    \end{itemize}
\end{itemize}

% operating on: pointer, read & write number of characters, seeking, end-of-file
While APIs may vary, the general model of interacting with a file is that the contents is an array of bytes, and when you open that file you get a point to the first element. You can then read or write (up to) some number of bytes from this position onwards. The pointer automatically advances, but it is also possible to perform arbitrary adjustments to this point by performing a \textsl{seek}\idx{Seek operation} operation. Reading beyond the largest index allocated in the file will result in reading the EOF\idx{End of file} (end-of-file) character.

% flushing

\subsection{Structure}

% directories: a special kind of file, a collection of files with unique names

\begin{figure}[tbp]
  \begin{center}
    \begin{tikzpicture}[remember picture]
      \genCoordsSDF{}
      \genFsI{}
    \end{tikzpicture}
  \end{center}
  \caption{Filesystem tree structure}
  \label{fig:topics:fs:dir}
\end{figure}

% mountpoints

\begin{figure}[tbp]
  \begin{center}
    \begin{tikzpicture}[remember picture]
      \genCoordsSDF{}
      \genFsI{}
      \genFsII{}
      \genMount{}
    \end{tikzpicture}
  \end{center}
  \caption{Mounting of one filesystem on another.}
  \label{fig:topics:fs:mount}
\end{figure}

% links: a named reference

\begin{figure}[tbp]
  \begin{center}
    \begin{tikzpicture}[remember picture]
      \genCoordsSDF{}
      \genFsI{}
      \genFsII{}
      \genMount{}
      \genLinkI{}
      \genLinkII{}
    \end{tikzpicture}
  \end{center}
  \caption[Filesystem links]{Filesystem links. Note that one link goes backward, thereby creating a cycle in the graph, while the other does not.}
  \label{fig:topics:fs:links}
\end{figure}

\subsection{Paths}

% absolute path

\begin{figure}[tbp]
  \begin{center}
    \begin{tikzpicture}[remember picture]
      \genCoordsSDF{}
      \genFsI{highlight:rel,highlight:abs}
      \genFsII{highlight}
      \genMount{highlight}
      \genLinkI{}
      \genLinkII{}
    \end{tikzpicture}
  \end{center}
  \caption{Absolute path.}
  \label{fig:topics:fs:path:abs}
\end{figure}

% relative path: context of a working directory

\begin{figure}[tbp]
  \begin{center}
    \begin{tikzpicture}[remember picture]
      \genCoordsSDF{}
      \genFsI{highlight:rel}
      \genFsII{highlight}
      \genMount{highlight}
      \genLinkI{}
      \genLinkII{}
      \genWD{}
    \end{tikzpicture}
  \end{center}
  \caption{Relative path.}
  \label{fig:topics:fs:path:rel}
\end{figure}

% qualified path

\subsection{Devices}

% what do you do with a file: open, close, read, write, seek

% devices are the same, and are (on a UNIX system) represented as files

\subsection{Inotify}

% problem of polling

% event subscription

\subsection{FUSE}

\subsection{Qualities}

\subsubsection{Snapshots}

\subsubsection{Transactional}

\subsection{Interaction Examples}

\subsubsection{Copying a File}

\begin{figure}[tbp]
  \begin{center}
    \begin{tikzpicture}
      \tikzstyle{title} = [
        anchor=south,
        align=center,
      ]
      \tikzstyle{arrow} = [
        thick,
        ->,
        >=stealth,
        draw=black,
      ]
      \tikzstyle{fileline}=[
        arrow,
        ultra thick,
        draw=black!20,
      ]
      \tikzstyle{fd} = [
        anchor=center,
        align=center,
        fill=black!20,
        minimum width=2cm,
        minimum height=0.6cm,
      ]
      \tikzstyle{data} = [
        anchor=center,
        fill=black,
        minimum width=1.5mm,
        minimum height=1.5mm,
      ]
      
      \newcommand{\shiftdistance}[0]{5cm}
      \newcommand{\figheight}[0]{5.2cm}
      
      \coordinate (memAnchor) at (0,-1cm);
      \coordinate (srcAnchor) at ([xshift=-\shiftdistance, yshift=0] memAnchor);
      \coordinate (dstAnchor) at ([xshift= \shiftdistance, yshift=0] memAnchor);
      
      % src line
      \node[title] (srcTitle) at ([yshift=2mm] srcAnchor) {Source\\File};
      \draw[fileline] (srcAnchor) -- ([yshift=-\figheight] srcAnchor);
      
      % dst line
      \node[title] (dstTitle) at ([yshift=2mm] dstAnchor) {Destination\\File};
      \draw[fileline] (dstAnchor) -- ([yshift=-\figheight] dstAnchor);
      
      % mem line
      \node[title] (memTitle) at ([yshift=2mm] memAnchor) {Program\\Memory};
      \draw[fileline] (memAnchor) -- ([yshift=-\figheight] memAnchor);
      
      % open(src)
      \node[fd] (srcOpen) at ([yshift=-4mm] srcAnchor) {\small{open file}};
      
      % open(dst)
      \node[fd] (dstOpen) at ([yshift=-11mm] dstAnchor) {\small{open fil}};
      
      % buffer
      \node[fd, minimum width=0.6cm, minimum height=1.4cm] (buffer) at ([yshift=-25mm] memAnchor) {\rotatebox{90}{\small{buffer}}};
      
      % read
      {
        \node[data] (srcDataI)   at ([yshift=-19mm] srcAnchor) {};
        \node[data] (srcDataII)  at ([yshift=-3.2mm] srcDataI) {};
        \node[data] (srcDataIII) at ([yshift=-3.2mm] srcDataII) {};
        
        \draw[arrow, dashed, draw=blue]
          ([xshift=4mm] srcDataII.center)--([xshift=\shiftdistance-0.3cm] srcDataII.center)
          node[midway,sloped,above]
          {\scalebox{0.7}{\textsl{\textcolor{blue}{read}}}};
      }
      
      % write
      {
        \node[data] (dstDataIII) at ([yshift=-31mm] dstAnchor) {};
        \node[data] (dstDataII)  at ([yshift=3.2mm] dstDataIII) {};
        \node[data] (dstDataI)   at ([yshift=3.2mm] dstDataII) {};
        
        \draw[arrow, dashed, draw=blue]
          ([xshift=-\shiftdistance+0.3cm] dstDataII.center)--([xshift=-4mm] dstDataII.center)
          node[midway,sloped,above]
          {\scalebox{0.7}{\textsl{\textcolor{blue}{write}}}};
      }
      
      % repetition
      \node[rectangle, thick, dashed, draw=purple, anchor=center, minimum width=2*\shiftdistance+10mm, minimum height=1.8cm] (repetition) at ([yshift=-25mm] memAnchor) {};
      \node[] (repetitionLabel) at ([xshift=-2mm] repetition.west) {\rotatebox{90}{\scalebox{0.60}{\textsl{\textcolor{purple}{repeat as needed}}}}};
      
      % close(src)
      \node[fd] (srcOpen) at ([yshift=-39mm] srcAnchor) {\small{close fil}};
      
      % close(dst)
      \node[fd] (dstOpen) at ([yshift=-46mm] dstAnchor) {\small{close fil}};
    \end{tikzpicture}
  \end{center}
  \caption{Flow followed when copying a file.}
  \label{fig:topics:fs:interaction:copy}
\end{figure}

\subsubsection{Editing a File}

Figure \ref{fig:topics:fs:interaction:edit} illustrates how a typical text editor with work with a file. First the file is opened. Then, the contents of the file is read and written to main memory. The user may then manipulate the in-memory representation through some user interface, and saved the modified buffer (memory representation) to disk. Finally, the program will close the file.

\begin{figure}[tbp]
  \begin{center}
    \begin{tikzpicture}
      \tikzstyle{title} = [
        anchor=south,
        align=center,
      ]
      \tikzstyle{arrow} = [
        thick,
        ->,
        >=stealth,
        draw=black,
      ]
      \tikzstyle{fileline}=[
        arrow,
        ultra thick,
        draw=black!20,
      ]
      \tikzstyle{fd} = [
        anchor=center,
        align=center,
        fill=black!20,
        minimum width=2cm,
        minimum height=0.6cm,
      ]
      \tikzstyle{data} = [
        anchor=center,
        fill=black,
        minimum width=1.5mm,
        minimum height=1.5mm,
      ]
      
      \newcommand{\shiftdistance}[0]{5cm}
      \newcommand{\moddistance}[0]{6mm}
      \newcommand{\figheight}[0]{5.2cm}
      
      \coordinate (memAnchor) at (0,-1cm);
      \coordinate (userAnchor) at ([xshift=-\shiftdistance, yshift=0] memAnchor);
      \coordinate (fileAnchor) at ([xshift= \shiftdistance, yshift=0] memAnchor);
      
      % user line
      \node[title] (userTitle) at ([yshift=2mm] userAnchor) {User\\Interface};
      \draw[fileline] (userAnchor) -- ([yshift=-\figheight] userAnchor);
      
      % file line
      \node[title] (fileTitle) at ([yshift=2mm] fileAnchor) {File of\\Interest};
      \draw[fileline] (fileAnchor) -- ([yshift=-\figheight] fileAnchor);
      
      % mem line
      \node[title] (memTitle) at ([yshift=2mm] memAnchor) {Program\\Memory};
      \draw[fileline] (memAnchor) -- ([yshift=-\figheight] memAnchor);
      
      % open(file)
      \node[fd] (fileOpen) at ([yshift=-4mm] fileAnchor) {\small{open file}};
      
      % buffer
      \node[fd, minimum width=0.6cm, minimum height=2.8cm] (buffer) at ([yshift=-25mm] memAnchor) {\rotatebox{90}{\small{buffer}}};
      
      % read
      {
        \node[data, anchor=north] (dataI)   at ([yshift=-2mm] fileOpen.south) {};
        \node[data] (dataII)  at ([yshift=-3.2mm] dataI) {};
        \node[data] (dataIII) at ([yshift=-3.2mm] dataII) {};
        
        \draw[arrow, dashed, draw=blue]
          ([xshift=-4mm] dataII.center)--([xshift=-\shiftdistance+0.3cm] dataII.center)
          node[midway,sloped,above]
          {\scalebox{0.7}{\textsl{\textcolor{blue}{read}}}};
      }
      
      % modify 1
      {
        \draw[arrow, dashed, draw=blue]
          ([xshift=-\shiftdistance, yshift=\moddistance] buffer.center)--([xshift=-0.3cm, yshift=\moddistance] buffer.center)
          node[midway,sloped,above]
          {\scalebox{0.7}{\textsl{\textcolor{blue}{modify}}}};
      }
      
      % modify 2
      {
        \draw[arrow, dashed, draw=blue]
          ([xshift=-\shiftdistance, yshift=0mm] buffer.center)--([xshift=-0.3cm, yshift=0mm] buffer.center)
          node[midway,sloped,above]
          {\scalebox{0.7}{\textsl{\textcolor{blue}{modify}}}};
      }
      
      % modify 3
      {
        \draw[arrow, dashed, draw=blue]
          ([xshift=-\shiftdistance, yshift=-\moddistance] buffer.center)--([xshift=-0.3cm, yshift=-\moddistance] buffer.center)
          node[midway,sloped,above]
          {\scalebox{0.7}{\textsl{\textcolor{blue}{modify}}}};
      }
      
      % write
      {
        \node[data] (dstDataIII) at ([yshift=-40mm] fileAnchor) {};
        \node[data] (dstDataII)  at ([yshift=3.2mm] dstDataIII) {};
        \node[data] (dstDataI)   at ([yshift=3.2mm] dstDataII) {};
        
        \draw[arrow, dashed, draw=blue]
          ([xshift=-\shiftdistance+0.3cm] dstDataII.center)--([xshift=-4mm] dstDataII.center)
          node[midway,sloped,above]
          {\scalebox{0.7}{\textsl{\textcolor{blue}{write}}}};
      }
      
      % close(file)
      \node[fd] (fileOpen) at ([yshift=-46mm] fileAnchor) {\small{close file}};
    \end{tikzpicture}
  \end{center}
  \caption{Flow followed by a text editor application.}
  \label{fig:topics:fs:interaction:edit}
\end{figure}

