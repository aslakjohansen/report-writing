\section{Graphs}

A graph\idx{Graph} is a structure that can be used to model how things relate to each other. Graphs consists of nodes\idx{Node} (also called vertices) and edges\idx{Edge} between them. There are many variations over the theme, but the most basic one is the \textsl{undirected}\idxx{Graph!Undirected}{Undirected} graph. It can be defined as:
\begin{eqnarray*}
  G = (N,E)
\end{eqnarray*}
Here, $N$ is the set of nodes and $E$ is the set of edges. We often choose to illustrate nodes as circles or boxes, and edges as lines.

% edges: directionality


\begin{figure}[tbp]
  \begin{center}
    \begin{tikzpicture}[remember picture]
      \newcommand{\weight}[1]{node[midway,sloped,above] {\scalebox{0.7}{\textsl{\textcolor{purple}{#1}}}}}
      \tikzstyle{edge}  = [thick,>=stealth,draw=black]
      \tikzstyle{dedge} = [thick,->,>=stealth,draw=black]
      \tikzstyle{node}=[
        overlay,
        circle,
        draw=purple,
        anchor=center,
        thick,
        minimum size=1,
      ]
      
      \node[node] (n1) at (-4,0) {$n_1$};
      \node[node] (n2) at (-1,-1) {$n_2$};
      \node[node] (n3) at (-3,-2) {$n_3$};
      \node[node] (n4) at (-1.7,-2.7) {$n_4$};
      \node[node] (n5) at (2.7,-2.5) {$n_5$};
      \node[node] (n6) at (3.8,-0.9) {$n_6$};
      \node[node] (n7) at (0.2,-2.2) {$n_7$};
      \node[node] (n8) at (1.3,-1.4) {$n_8$};
      \node[node] (n9) at (0.6,-0.2) {$n_9$};
      
      \draw[dedge] (n1)--(n2) \weight{2.3};
      \draw[dedge] (n2)--(n3) \weight{1.7};
      \draw[dedge] (n1)--(n3) \weight{1.8};
      \draw[dedge] (n2)--(n4) \weight{1.3};
      \draw[dedge] (n2)--(n9) \weight{1.5};
      \draw[dedge] (n9)--(n8) \weight{1.1};
      \draw[dedge] (n6)--(n9) \weight{2.4};
      \draw[dedge] (n5)--(n6) \weight{1.6};
      \draw[dedge] (n2)--(n7) \weight{1.3};
      \draw[dedge] (n7)--(n4) \weight{1.35};
      \draw[dedge] (n8)--(n5) \weight{1.4};
    \end{tikzpicture}
  \end{center}
  \caption{Example of directed graph with weighted edges.}
  \label{fig:topics:graphs:example:base}
\end{figure}

\subsection{Paths}

\begin{figure}[tbp]
  \begin{center}
    \begin{tikzpicture}[remember picture]
      \newcommand{\weight}[1]{node[midway,sloped,above] {\scalebox{0.7}{\textsl{\textcolor{purple}{#1}}}}}
      \tikzstyle{edge}  = [>=stealth,draw=black]
      \tikzstyle{dedge} = [->,>=stealth,draw=black]
      \tikzstyle{node}=[
        overlay,
        circle,
        draw=purple,
        anchor=center,
        minimum size=1,
      ]
      
      \node[node, very thick, fill=purple!10] (n1) at (-4,0) {$n_1$};
      \node[node, very thick, fill=purple!10] (n2) at (-1,-1) {$n_2$};
      \node[node] (n3) at (-3,-2) {$n_3$};
      \node[node] (n4) at (-1.7,-2.7) {$n_4$};
      \node[node, very thick, fill=purple!10] (n5) at (2.7,-2.5) {$n_5$};
      \node[node] (n6) at (3.8,-0.9) {$n_6$};
      \node[node] (n7) at (0.2,-2.2) {$n_7$};
      \node[node, very thick, fill=purple!10] (n8) at (1.3,-1.4) {$n_8$};
      \node[node, very thick, fill=purple!10] (n9) at (0.6,-0.2) {$n_9$};
      
      \draw[dedge, very thick] (n1)--(n2) \weight{\textbf{2.3}};
      \draw[dedge] (n2)--(n3) \weight{1.7};
      \draw[dedge] (n1)--(n3) \weight{1.8};
      \draw[dedge] (n2)--(n4) \weight{1.3};
      \draw[dedge, very thick] (n2)--(n9) \weight{\textbf{1.5}};
      \draw[dedge, very thick] (n9)--(n8) \weight{\textbf{1.1}};
      \draw[dedge] (n6)--(n9) \weight{2.4};
      \draw[dedge] (n5)--(n6) \weight{1.6};
      \draw[dedge] (n2)--(n7) \weight{1.3};
      \draw[dedge] (n7)--(n4) \weight{1.35};
      \draw[dedge, very thick] (n8)--(n5) \weight{\textbf{1.4}};
    \end{tikzpicture}
  \end{center}
  \caption[Example of a path in a directed graph.]{Example of a path in a directed graph. The path starts in $n_1$, goes through nodes $n_2$, $n_9$, $n_8$ to end in $n_5$. It has a total length of 6.3.}
  \label{fig:topics:graphs:example:path}
\end{figure}

\subsection{Cycles}

\begin{figure}[tbp]
  \begin{center}
    \begin{tikzpicture}[remember picture]
      \newcommand{\weight}[1]{node[midway,sloped,above] {\scalebox{0.7}{\textsl{\textcolor{purple}{#1}}}}}
      \tikzstyle{edge}  = [>=stealth,draw=black]
      \tikzstyle{dedge} = [->,>=stealth,draw=black]
      \tikzstyle{node}=[
        overlay,
        circle,
        draw=purple,
        anchor=center,
        minimum size=1,
      ]
      
      \node[node] (n1) at (-4,0) {$n_1$};
      \node[node] (n2) at (-1,-1) {$n_2$};
      \node[node] (n3) at (-3,-2) {$n_3$};
      \node[node] (n4) at (-1.7,-2.7) {$n_4$};
      \node[node, very thick, fill=purple!10] (n5) at (2.7,-2.5) {$n_5$};
      \node[node, very thick, fill=purple!10] (n6) at (3.8,-0.9) {$n_6$};
      \node[node] (n7) at (0.2,-2.2) {$n_7$};
      \node[node, very thick, fill=purple!10] (n8) at (1.3,-1.4) {$n_8$};
      \node[node, very thick, fill=purple!10] (n9) at (0.6,-0.2) {$n_9$};
      
      \draw[dedge] (n1)--(n2) \weight{2.3};
      \draw[dedge] (n2)--(n3) \weight{1.7};
      \draw[dedge] (n1)--(n3) \weight{1.8};
      \draw[dedge] (n2)--(n4) \weight{1.3};
      \draw[dedge] (n2)--(n9) \weight{1.5};
      \draw[dedge, very thick] (n9)--(n8) \weight{\textbf{1.1}};
      \draw[dedge, very thick] (n6)--(n9) \weight{\textbf{2.4}};
      \draw[dedge, very thick] (n5)--(n6) \weight{\textbf{1.6}};
      \draw[dedge] (n2)--(n7) \weight{1.3};
      \draw[dedge] (n7)--(n4) \weight{1.35};
      \draw[dedge, very thick] (n8)--(n5) \weight{\textbf{1.4}};
    \end{tikzpicture}
  \end{center}
  \caption[Example of a cycle in a directed graph.]{Example of a cycle in a directed graph. It goes through nodes $n_9$, $n_8$, $n_5$, and $n_6$.}
  
  \label{fig:topics:graphs:example:cycle}
\end{figure}

\subsection{Directed Acyclic Graphs}

\subsection{Trees}

\subsection{Connectedness}

\subsection{Reachability}

\subsection{Representation}
\subsubsection{Marshalling}
\subsubsection{Datastructures}

\subsection{Property Graphs}

\subsection{Visual Layout}

\subsection{Use Cases}

\begin{itemize}
  \descitem{Filesystem}
  \descitem{GIT}
  \descitem{User Interfaces}
  \descitem{Dependency Modeling}
  \descitem{HTML/XML}
  \descitem{Route Planning}
\end{itemize}

