\section{Versioning Control Systems}

% what

% git focus: there are many cvs's, git is part of a "new" generation and by far the most popular of those, unless you have a good reason you should use git for every new project
Over the years, there has been many versioning control systems in use. CVS\idx{CVS} has critical issues and should absolutely not be used today. It was superceded by subversion\idx{Subversion} (SVN) that fixed all those issues. After subversion came a new generation of \textsl{distributed} versionsing control systems (e.g., GIT, mercurial\idx{Mercurial} and monotone\idx{Monotone}). Among these, GIT\idx{GIT} is by far the most popular one. Unless you have a really good reason you should use GIT for every new project. And for that reason, GIT is what is being covered here.

\subsection{Diffs}

See figure \ref{topics:vcs:diff}.

\begin{figure}[tbp]
  \vspace{10mm}
  \begin{tikzpicture}[remember picture, overlay]
    % variables
    \newcommand{\spacing}[0]{4mm}
    
    \coordinate (origo) at (40mm,0);
    
    % styles
    \tikzstyle{file}  = [
      rectangle,
      draw,
      anchor=west,
      align=left,
      scale=0.71,
    ]
    
    
    \node[file,rounded corners,draw=teal,fill=teal!20] (diff) at (origo) {
        \texttt{3c3}\\
        \texttt{< I hope you are doing wel.}\\
        \texttt{---}\\
        \texttt{> I hope you are doing well.}
    };
    \draw[draw=teal] (diff.south) -- ([yshift=-\spacing] diff.south);
    \draw[->,>=stealth,draw=teal] ([xshift=-\spacing,yshift=-\spacing] diff.south west) -- ([xshift=\spacing,yshift=-\spacing] diff.south east);
    
    \node[file, anchor=east] (input) at ([xshift=-\spacing,yshift=-\spacing] diff.south west) {
      \small
      \texttt{Hello Mom,}\\
      \texttt{}\\
      \texttt{I hope you are doing wel.}\\
      \texttt{}\\
      \texttt{Best,}\\
      \texttt{Billie}
    };
    \node[file, anchor=west] (output) at ([xshift=\spacing,yshift=-\spacing] diff.south east) {
      \small
      \texttt{Hello Mom,}\\
      \texttt{}\\
      \texttt{I hope you are doing well.}\\
      \texttt{}\\
      \texttt{Best,}\\
      \texttt{Billie}
    };
  \end{tikzpicture}
  \vspace{22mm}
  
  \caption{Coding of the difference between two revisions of the same file.}
  \label{topics:vcs:diff}
\end{figure}

\subsection{Commit Logs}

\subsubsection{State Machine}

\begin{figure}[tbp]
  \begin{tikzpicture}[]
    % variables
    \newcommand{\spacing}[0]{20mm}
    \newcommand{\statesize}[0]{20mm}
    \newcommand{\radius}[0]{24mm}
    \newcommand{\groupsize}[0]{(2*\radius+\statesize+\spacing/2)}
    
    \coordinate (origo) at (80mm,-\spacing-1.5*\statesize);
    
    % styles
    \tikzstyle{state}  = [
      circle,
      draw,
      anchor=center,
      align=left,
      minimum width=\statesize,
      scale=1.0,
    ]
    \tikzstyle{group}  = [
      rectangle,
      draw=purple,
      rounded corners,
      fill=purple!10!white,
      anchor=center,
      minimum width=\groupsize,
      minimum height=\groupsize,
      scale=1.0,
    ]
    \tikzstyle{arrow}  = [
      ->,
      >=stealth,
      draw,
    ]
    
    \node[group] (group) at (origo) {};
    \node[text=purple] (grouplabel) at ([yshift=-0.5*\spacing] group.north) {tracked};
    
    \node[state] (staged)    at ([xshift=-0.8660254037844387*\radius, yshift=0.49999999999999994*\radius] origo) {staged};
    \node[state] (comitted)  at ([xshift=0.8660254037844387*\radius, yshift=0.49999999999999994*\radius] origo) {comitted};
    \node[state] (modified)  at ([yshift=-\radius] origo) {modified};
    \node[state, anchor=east] (untracked) at ([xshift=-1.5*\spacing] staged.west) {untracked};
    
    \draw[arrow] (untracked) -- (staged) node[midway,sloped,above] {add};
    \draw[arrow] (staged) -- (comitted) node[midway,sloped,above] {commit};
    \draw[arrow] (comitted) -- (modified) node[midway,sloped,above] {edit};
    \draw[arrow] (modified) -- (staged) node[midway,sloped,above] {stage};
    \draw[arrow,draw=purple,text=purple] ([yshift=-1/6*\groupsize] group.west) -- (untracked) node[midway,sloped,above] {remove};
  \end{tikzpicture}
  \caption{File status lifecycle in GIT.}
  \label{topics:vcs:file:status}
\end{figure}

% typical workflow
A typical workflow for implementing a change involves:
\begin{enumerate}
  \descitem{Update} Modify committed files and add new (and thus untracked) files.
  \descitem{Select} Select files whose changes represent a \quoted{unit} of change. This is usually all the updated files, but not always. Good qualities of a unit of change is that
    \begin{itemize}
      \item It can be described briefly and succinctly.
      \item It represents a change that is easy to reasons about. It being easy to reason about means that you can easily think about the codebase both with and without the change (even if you have multiple later commits). This allows you to later remove that commit from the timeline of the codebase with confidence, should it turn out to have negative sideeffects.
    \end{itemize}
  \descitem{Commit} Commit the selected files.
\end{enumerate}

\subsubsection{Status}

\subsection{Branches}

\subsection{Merging}

