\chapter{\LaTeX}
\label{latex}

\section{Base Document}

\subsection{Report}
\subsection{Presentation}

\idx{Presentation}There are a number of styles for creating \LaTeX presentations. I have always used the \texttt{beamer} package\idxx{beamer package@\texttt{beamer} package}{\texttt{beamer} package}.

\scalebox{0.96}{\url{https://en.wikibooks.org/wiki/LaTeX/Presentations\#The_Beamer_package}}

\subsection{Poster}

\idx{Poster}There are a few options for creating posters in \LaTeX. I am currently using the \texttt{tikzposter} package\idxx{tikzposter package@\texttt{tikzposter} package}{\texttt{tikzposter} package}. An example can be found here:

\scalebox{0.96}{\url{https://github.com/aslakjohansen/poster-sample/}}

\subsection{Commands and Environments}

% commands

% environments

\section{Basic Structure}
\subsection{Sections}

% levels

% sepecial case of chapters

\subsection{Paragraphs}

\section{Basic Formatting}

\section{Lists}
\subsection{Bullet Points}

\idxx{Itemize environment@\texttt{itemize} environment}{\texttt{itemize} environment}
Bulletpoints are coded using the \texttt{itemize} environment, and are usefull for listing the entries of a set.

\begin{latexdemo}
  \begin{minted}[breaklines]{latex}
  \begin{itemize}
    \item Bonobo
    \item Gorilla
    \item Gibbon
  \end{itemize}
  \end{minted}
  &
  \begin{itemize}
    \item Bonobo
    \item Gorilla
    \item Gibbon
  \end{itemize}
\end{latexdemo}

\subsection{Enumerated Points}

\idxx{Enumerate environment@\texttt{enumerate} environment}{\texttt{enumerate} environment}
\begin{latexdemo}
  \begin{minted}[breaklines]{latex}
  \begin{enumerate}
    \item Processor
    \item Memory
    \item Network
  \end{enumerate}
  \end{minted}
  &
  \begin{enumerate}
    \item Processor
    \item Memory
    \item Network
  \end{enumerate}
\end{latexdemo}

\subsection{Nesting}

\section{Floats}

\section{Appendices}

\section{Transformations}
\subsection{Scale}
\subsection{Rotation}

\section{Bibliography}
\section{Table of Contents}
\section{Index}

\section{Code Inclusion}

\section{Graphics}
\subsection{Vector Graphic with Ti\textit{k}Z}

\idxx{Tikz@Ti\textit{k}Z}{Ti\textit{k}Z}
Sources:
\begin{itemize}
  \item \url{https://tikz.dev}
  \item \url{https://en.wikibooks.org/wiki/LaTeX/PGF/TikZ}
\end{itemize}

\subsection{External Vector Graphics}
\subsection{External Bitmapped Graphcs}

\section{Macros}

% importance of building abstractions

\subsection{Examples}

\subsection{Color Themes}
\subsection{Highlights}
\subsection{Description Items}

I often want a heading for each \texttt{item} in an \texttt{itemize} environment. This is convenient for descriptions. So, for convenience I define a \texttt{\textbackslash descitem}\idxx{Descitem command@\texttt{\textbackslash descitem} command}{\texttt{\textbackslash descitem} command} (short for description item) command that calls \texttt{item} and highlights a text. To be able to refer to that text using the same highlight I also define a \texttt{\textbackslash textdesc}\idxx{Textdesc command@\texttt{\textbackslash textdesc} command}{\texttt{\textbackslash textdesc} command} command.

\begin{latexdemo}
  \begin{minted}[breaklines,fontsize=\footnotesize]{latex}
  \newcommand{\textdesc}[1]{
    \textit{\textbf{#1}}
  }
  \newcommand{\descitem}[1]{
    \item \textdesc{#1}
  }
  
  \begin{itemize}
    \descitem{Bonobo} Clever ape.
    \descitem{Gorilla} Strong ape.
    \descitem{Gibbon} Agile ape.
  \end{itemize}
  
  I like the \textdesc{Bonobo}!
  \end{minted}
  &
  \begin{itemize}
    \descitem{Bonobo} Clever ape.
    \descitem{Gorilla} Strong ape.
    \descitem{Gibbon} Agile ape.
  \end{itemize}
  
  I like the \textdesc{Bonobo}!
\end{latexdemo}

