\chapter{\LaTeX}
\label{latex}

\section{Base Document}

\subsection{Report}
\subsection{Presentation}
\subsection{Poster}

\section{Basic Structure}
\subsection{Sections}
\subsection{Paragraphs}

\section{Basic Formatting}

\section{Lists}
\subsection{Bullet Points}

Bulletpoints are coded using the \texttt{itemize} environment, and are usefull for listing the entries of a set.

\begin{tabular}{|p{0.46\linewidth}|p{0.46\linewidth}|}
  \hline
  \textbf{Code:} & \textbf{Outcome:} \\
  \hline
  \begin{minted}[breaklines]{latex}
  \begin{itemize}
    \item Bonobo
    \item Gorilla
    \item Gibbon
  \end{itemize}
  \end{minted}
  &
  \begin{itemize}
    \item Bonobo
    \item Gorilla
    \item Gibbon
  \end{itemize}
  \\
  \hline
\end{tabular}

\subsection{Enumerated Points}
\subsection{Nesting}

\section{Floats}

\section{Appendices}

\section{Transformations}
\subsection{Scale}
\subsection{Rotation}

\section{Bibliography}
\section{Table of Contents}
\section{Index}

\section{Code Inclusion}

\section{Graphics}
\subsection{Vector Graphic with Ti\textit{k}Z}
\subsection{External Vector Graphics}
\subsection{External Bitmapped Graphcs}

\section{Macros}

% importance of building abstractions

\subsection{Examples}

\subsection{Color Themes}
\subsection{Highlights}
\subsection{Description Items}

